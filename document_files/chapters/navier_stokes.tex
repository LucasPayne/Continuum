% The Navier-Stokes equations

% \section{The kinematics of flow}
% \subsection{Vorticity}
% \subsection{The Helmholtz decomposition}
% \subsubsection{The stream function}
% \subsubsection{The velocity potential}

\section{Introduction}
% <<<
The incompressible Navier-Stokes equations model the motion of a common kind of viscous fluid called a \textit{Newtonian fluid}.
They are:
\begin{itemize}
\item The Cauchy momentum equation \eqref{cauchy_continuity_differential_material} for constant mass density $\rho$ and velocity $u$,
\item an incompressibility constraint $\nabla\cdot u = 0$ and unknown pressure $p$,
\item and a concrete constitutive relation for the deviatoric stress $\tau$.
\end{itemize}
In anticipation, their common form is
\begin{equation}\label{navier_stokes}
    \rho\frac{Du}{Dt} = - \nabla p + \nabla\cdot\tau + \rho g,\quad \nabla\cdot u = 0,
\end{equation}
where $\tau$ is defined by Stokes' constitutive relation
\begin{equation}\label{stokes_constitutive_relation}
    \tau = \mu\left(\nabla u + \nabla u^T\right),
\end{equation}
where $\mu$ is called the \textit{viscosity}, and $\nabla u$ is the velocity gradient defined in section \ref{deformation_and_velocity_gradients},
measuring the local deformation of a small control volume under
the flow of $u$. We will assume their domain is a subset of $\mathbb{R}^d$, where typically $d = 2$ or $3$, although the Navier-Stokes equations can be solved
in curved domains (see \cite{stam}).
Alongside a domain and appropriate initial and boundary conditions, the Navier-Stokes equations \eqref{navier_stokes}
form a concrete flow problem which
can be solved numerically, or in special situations analytically.
% >>>

\section{The equations of fluid motion}
% <<<
% Introduction
% <<<
We begin with the standard conservation equations of mass and linear momentum, \eqref{conservation_system}:
\begin{equation}\label{compressible_system}
\begin{split}
    \frac{D\rho}{Dt} + \rho\nabla\cdot u = 0 &\quad\text{(Conservation of mass)},
    \\
    \rho\frac{Du}{Dt} = \rho g + \nabla\cdot\sigma &\quad\text{(Conservation of linear momentum)}.
\end{split}
\end{equation}
% To add incompressibility, our first step is to change the conservation of mass equation to
% \begin{equation*}
%     \nabla \cdot u = 0.
% \end{equation*}
% Assuming a constant mass density $\rho$, this also expresses mass conservation. This will preclude phenomena such as acoustic waves in the fluid,
% but is a good approximation for many studies of fluid motion (see Batchelor \cite{batchelor}).
% However we now have a problem: The conservation of linear momentum in \eqref{compressible_system} has a source term $\rho g + \nabla\cdot\sigma$
% \newcommand{\virtualforce}{{F_{\text{virtual}}}}
% which needs not be non-compressing. We need to introduce some virtual force $\virtualforce$ such that
%     $$\rho g + \nabla\cdot\sigma + \virtualforce$$
% such that is non-compressing
As discussed in section \ref{constitutive_relations}, we need to specify the Cauchy stress tensor $\sigma$ such that the system is well-formed.
It would be helpful to restrict the possible
form of $\sigma$. In the next section we will show that $\sigma$'s antisymmetric part describes a couple force, a force which induces an angular
momentum (a ``spin'') in a small control volume, but which doesn't contribute to linear momentum. Therefore, if we do not want couple forces,
we want $\sigma$ to be symmetric.
% >>>
\subsection{Conservation of angular momentum}
% <<<
Angular momentum is traditionally presented in terms of rigid bodies, bodies subject to a distance-preserving constraint between
material points. It is the moment of linear momentum.
The discussion in section \ref{material_points} indicates that we can think of a very small rigid body at a material point, subject to the flow.
This will be subject to a ``spin force''.

Let the material point be $c \in \mathbb{R}^d$, which will act as a ``centre of mass', and let $c \in \Omega_0$.
Define $\bar{x} \coloneqq x - c$.
Define the moment of linear momentum as
\begin{equation}\label{moment_of_linear_momentum}
    \int_{\omn} \bar{x} \wedge \left(\rho u\right)\,dx.
\end{equation}
The symbol $\wedge$ indicates the cross product, whose value should be thought of as a pseudo-vector or ``plane with magnitude''.
We will call this the angular momentum of the control volume $\omn$.
We repeat here an integral form of linear momentum conservation, which already must hold:
\begin{equation}\label{linear_momentum_for_angular}
    \int_{\omn} \Part{(\rho u)}{t}\,dx + \int_{\pomn} \rho u (u\cdot \hat{n})\,dx
    = \int_{\omn} \rho g\,dx + \int_{\pomn} \sigma \hat{n}\,dx.
\end{equation}
It is simple to derive an angular momentum conservation equation, just by taking moments of each vector quantity:
\begin{equation}\label{moment_of_linear_momentum_sources}
    \int_{\omn} \bar{x} \wedge \Part{(\rho u)}{t}\,dx + \int_{\pomn} \bar{x} \wedge \left(\rho u\right)\left(u\cdot \hat{n}\right)\,dx
    = \int_{\omn} \bar{x} \wedge \left(\rho g\right)\,dx + \int_{\pomn} \bar{x} \wedge \left(\sigma \hat{n}\right)\,dx.
\end{equation}
(This precludes the introduction of surface and body couples which induce no linear momentum but do induce angular momentum
\cite{leal}. We ignore these torques.)
\\
If \eqref{linear_momentum_for_angular} holds, should \eqref{moment_of_linear_momentum_sources} hold automatically?
% Note that as $\Omega_0$ becomes small, $\hat{x}$ will become approximately parallel to $\hat{n}$.
% 
By Stokes' theorem, the final term in \eqref{linear_momentum_for_angular} can be written as
\begin{align*}
    \int_{\pomn} \sigma\hat{n}\,dx = \int_\omn\nabla\cdot\sigma\,dx.
\end{align*}
% \vskip 0.2in
% (draw the differentialization happening at each point, contracting a small control volume).
% \vskip 0.2in
% With the help of the above picture,
We can try to derive a per-point form for the final term in \eqref{moment_of_linear_momentum_sources},
\begin{align*}
    \int_{\pomn} \bar{x} \wedge \left(\sigma\hat{n}\right)\,dx = \int_\omn\cdots ? \cdots\,dx.
\end{align*}
At a point $c$ in $\omn$, we contract an even smaller control volume $\om_c$ around that point, in order to express
the boundary integral over $\pomn$ in terms of smaller boundary integrals on the interior. As this takes a limit,
we can separately assume that $\bar{x} = x - c$ and $\sigma$ are constant in $\om_c$, giving the ``product rule''
\begin{align*}
    \int_{\pom_c} \bar{x}\wedge\left(\sigma\hat{n}\right)\,dx \quad\rightarrow\quad
    \bar{x}\wedge \nabla\cdot \sigma + \text{(some term keeping $\sigma$ constant)}.
\end{align*}
We can reason geometrically to find the final term. We can split $\sigma$ into its symmetric part $S$ and antisymmetric part $N$:
    $$\sigma = \frac{1}{2}\left(\sigma + \sigma^T\right) + \frac{1}{2}\left(\sigma - \sigma^T\right) = S + N.$$
Antisymmetric matrices have a lot to do with rotations: A special property of antisymmetric matrices is
    $$\inner{x, Nx} = \inner{x, N^Tx} = \inner{x, -Nx} \quad\Rightarrow\quad \inner{x, Nx} = 0.$$
In fact the antisymmetric matrices are exactly those which generate rotations (formally, $\exp(N)$ is orthogonal, where $\exp$ is the matrix
exponential). If $\sigma$ is kept constant over $\pom_c$, we can consider the tractions contributed by the symmetric and antisymmetric parts of $\sigma$,
concluding that, letting $\sigma = S + N$ be constant,
\begin{align*}
    \int_{\pom_c} \bar{x} \wedge \left(\left(S + N\right)\hat{n}\right)\,dx
    =
    \int_{\pom_c} \bar{x} \wedge \left(N\hat{n}\right)\,dx
    \quad\rightarrow\quad
    \widehat{N}
\end{align*}
where $\widehat{N}$ is defined in $\mathbb{R}^3$ as the axis-angle vector representation of the differential rotation corresponding to $N$:
\begin{equation}
\begin{split}
    Nv = \widehat{N}\wedge v,\quad v\in\mathbb{R}^3
    \\
    N = \begin{bmatrix}
        0 & -\omega_z & \omega_y \\ \omega_z & 0 & -\omega_x  \\ -\omega_y & \omega_x & 0
    \end{bmatrix} \Rightarrow \widehat{N} = \begin{bmatrix} \omega_x \\ \omega_y \\ \omega_z \end{bmatrix}.
\end{split}
\end{equation}
We now have
\begin{equation}
    \int_{\pom_c} \bar{x}\wedge\left(\sigma\hat{n}\right)\,dx \quad\rightarrow\quad
    \bar{x}\wedge \nabla\cdot \sigma + \widehat{N},
\end{equation}
which gives 
\begin{equation}
    \int_{\pomn} \bar{x} \wedge \left(\sigma\hat{n}\right)\,dx = \int_\omn \bar{x}\wedge\nabla\cdot\sigma + \widehat{N}\,dx.
\end{equation}
This shows that for the tractions measured across the boundary, although their contributions to the linear momentum of the control volume conserve it,
there is another contribution to the angular momentum, which is the ``spin part'' of $\sigma$.
To show this more directly, localise the linear conservation equation \eqref{linear_momentum_for_angular}:
\begin{equation}\label{linear_momentum_for_angular_localised}
    \int_{\omn} \Part{(\rho u)}{t} + \nabla\cdot\left(\rho u\otimes u\right) - \rho g - \nabla\cdot\sigma\,dx = 0.
\end{equation}
The corresponding differential form of \eqref{moment_of_linear_momentum_sources} is
\begin{equation}\label{moment_of_linear_momentum_sources_localised}
    \int_{\omn} \bar{x} \wedge \left[\Part{(\rho u)}{t} + \nabla\cdot\left(\rho u\otimes u\right) - \rho g - \nabla\cdot\sigma\right]\,dx
    = \int_\omn \widehat{N}\,dx.
\end{equation}
As the conservation law \eqref{linear_momentum_for_angular_localised} must hold for all $\Omega_0$, we can see that for \eqref{linear_momentum_for_angular_localised}
to imply \eqref{moment_of_linear_momentum_sources_localised} (for linear momentum conservation to imply angular momentum conservation) we need
\begin{equation}
    \widehat{N} = 0 \quad\Rightarrow\quad \sigma = \sigma^T.
\end{equation}
So, without the introduction of any explicit angular momentum sources (body and surface couples), the Cauchy stress tensor $\sigma$ is required to be symmetric,
reducing the $d^2$ unknowns to $d(d+1)/2$ unknowns.
% (~~~ May be $\widehat{N}/2$, maybe a sign error. Check Leal p68.)
% (... figure)
% \begin{center}
% \includegraphics[page=1,width=0.7\linewidth]{figures/3.pdf}
% \end{center}


% If the control volume follows the flow, the Reynolds transport theorem \eqref{reynolds_transport_theorem} gives a rate of change of angular momentum:
% \begin{equation}\label{moment_of_linear_momentum_reynolds}
%     \frac{d}{dt}\eval{\left[\int_{\omn(t)} \bar{x} \wedge \left(\rho u\right)\,dx\right]}_{t=0}
%     % = \int_{\omn(0)} \left(x - c\right)\wedge \left(\rho g\right)\,dx + \int_{\pomn(0)} \left(x - c\right) \wedge \hat
%     = \int_{\omn(0)} \bar{x} \wedge \Part{(\rho u)}{t}\,dx + \int_{\pomn(0)} \bar{x} \wedge \rho u (u\cdot \hat{n})\,dx.
% \end{equation}
% The source terms of linear momentum are body forces $\rho g$ on the interior and tractions $\hat{t}$ on the boundary.
% Therefore we have an explicit equation for the rate of change of angular momentum, which we equate to the right-hand-side of
% \eqref{moment_of_linear_momentum_reynolds}:
% \begin{equation}\label{moment_of_linear_momentum_sources}
%     \int_{\omn(0)} \bar{x} \wedge \Part{(\rho u)}{t}\,dx + \int_{\pomn(0)} \bar{x} \wedge \rho u (u\cdot \hat{n})\,dx
%     = \int_{\omn(0)} \bar{x} \wedge \left(\rho g\right)\,dx + \int_{\pomn(0)} \bar{x} \wedge \left(\sigma \hat{n}\right)\,dx.
% \end{equation}
% By the Euler-Cauchy stress principle (section \ref{stress_principle}), we have let the traction $\hat{t} = \sigma \hat{n}$.
% For comparison, we repeat here an integral form of linear momentum conservation, which already must hold:
% \begin{equation*}
%     \int_{\omn(0)} \Part{(\rho u)}{t}\,dx + \int_{\pomn(0)} \rho u (u\cdot \hat{n})\,dx
%     = \int_{\omn(0)} \left(\rho g\right)\,dx + \int_{\pomn(0)} \sigma \hat{n}\,dx.
% \end{equation*}
% >>>
% >>>
% \subsection{Conservation of energy}
% % <<<
% \begin{equation*}
% \begin{split}
%     \frac{D\rho}{Dt} + \rho\nabla\cdot u = 0 &\quad\text{(Conservation of mass)},\\
%     \rho\frac{Du}{Dt} = \rho g + \nabla\cdot\sigma &\quad\text{(Conservation of linear momentum)},\\
%     \sigma = \sigma^T &\quad\text{(Conservation of angular momentum)}.
% \end{split}
% \end{equation*}
% % >>>

\section{The Navier-Stokes constitutive relation}
We can keep going from here, and derive restrictions on possible constitutive relations due to, for example, conservation of energy.
This leads to general classes of physical fluid models, including some ``memoryless'' complex fluids. Typically,
new parameters are then introduced which may be constant, or themselves determined by the state of the continuum model, such as
bulk viscosity, pressure, and temperature.
The incompressible Navier-Stokes equations are one of the simplest such physically realistic fluid models,
introducing one parameter (the shear viscosity $\mu$) and one unknown (the pressure $p$).
\subsection{The Euler equations}
Consider letting $\sigma = -pI$, where $p$ is a new unknown, to be interpreted as pressure.
Combining the momentum equation of \eqref{compressible_system} with a volume preserving constraint $\nabla\cdot u = 0$,
determines the unknown $p$ completely. $p$ is then interpreted as determining a conservative, isotropic force on material elements, which
``pushes them apart'' in order to obey the new constraint. The pressure will soon be discussed in some detail.
This constitutive relation gives the \textit{Euler equations}, modelling an entirely inviscid incompressible fluid. Although this is perhaps
the simplest fluid model to write down, the resulting equations,
\begin{equation}\label{euler_equations}
    \rho\frac{Du}{Dt} = - \nabla p + \rho g,\quad \nabla\cdot u = 0,
\end{equation}
are hyperbolic and very difficult to solve. An entirely inviscid fluid is also not very physically realistic --- a small amount of
viscosity (resistence to shear) can resolve the resulting issues at boundary interfaces \cite{batchelor} \cite{leal}.

\subsection{Introducing viscosity forces}
The argument here is somewhat heuristic.
The viscosity force is intended to give shear resistance to a fluid, and the deformation gradient $\nabla u$ gives local information about
this shear. We can consider the separate components of $u$, say for $d = 2$, being $u_x$ and $u_y$.
Consider two small control volumes meeting at an interface aligned to the $y$-axis, and a laminar flow moving
only in the $y$-axis. If $\Part{u_y}{x}$ is positive
at the interface, this means that there is shear. The viscosity force is expected to cause
the bulk velocities of these adjacent control volumes to tend to the same velocity near the interface.
The positive $\Part{u_y}{x}$ term then indicates a positive force at the interface, which is increasing
the speed of the slower left control volume (near the interface), and decreasing the speed of the faster right control volume.
The idea of the Navier-Stokes relation is that a vector \textit{diffusion} models this effect nicely. We can let
    $$\sigma = -pI + \mu \nabla u,$$
where $\mu > 0$ is a new parameter which determines the speed of the momentum diffusion.
The resulting equations of fluid motion, in differential form, applying the divergence theorem to the Cauchy stress tensor $\sigma$'s
contraction with the boundary elements (computing the traction forces), are
\begin{equation}\label{navier_stokes_final}
    \rho\left(\Part{u}{t} + u\cdot \nabla u\right) = - \nabla p + \mu\Delta u + \rho g,\quad \nabla\cdot u = 0.
\end{equation}
The viscosity term $\mu\Delta u$ is linear, and the resulting equations, the incompressible Navier-Stokes equations are parabolic and
non-linear, due to the advection term expanded from $\rho\frac{Du}{Dt}$.
The unknowns are the velocity field $u$, and the pressure $p$.

\subsection{The Reynolds number}
Even with viscosity preventing discontinuities in a flow, the Navier-Stokes equations
still exhibit \textit{turbulence}. The patterns of motion resulting from even simple initial conditions and sources of momentum
can be as intricate as the motion of the real fluids they model -- whirling, bifurcating, and colliding vortices, unpredictable waving
and wisps of velocity fluctuations, and so on. However, if viscosity is high enough with respect
to some choice of reference measurements of the domain and average fluid speed, then the non-linear advective term $\rho u\cdot \nabla u$
in \eqref{navier_stokes_final}, which is both the origin of nonlinearity and of these complex flow patterns, can be ignored.
The most common value used in determining the possible complexity of the flow is the non-dimensional \textit{Reynolds number},
    $$Re = \frac{\rho \norm{u} L}{\mu}.$$
$L$ is a reference length, such as the length of a pipe domain or the diameter of an obstructing circle of interest. $\norm{u}$ can be
taken as the average speed of fluid motion. $\mu$ is the shear viscosity parameter of the Navier-Stokes equations,
which measures the strength of momentum diffusion. Therefore,
    $$\text{A low enough Reynolds number} \quad\Rightarrow\quad \text{Momentum diffusion has a much greater effect than momentum advection.}$$
We will firstly consider approximations of the Navier-Stokes equations \eqref{navier_stokes_full} for very low Reynolds number flow.

\section{Stokes flow and the meaning of pressure}\label{pressure_derivation}
% <<<
If we assume that the advective term $u\cdot \nabla u$ in the incompressible Navier-Stokes equations \eqref{navier_stokes_final} is ``small'',
we can ignore it and derive the linear \textit{unsteady Stokes equations}:
\begin{equation}\label{unsteady_stokes}
    \rho\Part{u}{t} = \mu\Delta u + \rho g - \nabla p, \quad \nabla\cdot u = 0.
\end{equation}
We are assuming validity for low Reynolds number $Re \ll 1$, where convective behaviour is neglible compared to the viscous forces, which for a
Navier-Stokes fluid ``diffuse'' the linear momentum. Setting the left-hand-side of \eqref{unsteady_stokes} to zero results in the \textit{steady Stokes equations}
\begin{equation}\label{steady_stokes}
    \mu\Delta u + \rho g - \nabla p = 0,\quad \nabla\cdot u = 0.
\end{equation}
Time-dependent equation \eqref{unsteady_stokes} can be thought of as a ``gradient descent'' to find the steady Stokes flow \eqref{steady_stokes}.
The steady Stokes equation is a constrained vector Poisson equation, where we have introduced pressure $p$ explicitly.
It is well-known, by Dirichlet's principle, that we can think
of a weak solution to the unconstrained vector Poisson equation as a minimiser of the Dirichlet energy,
\begin{equation}
\begin{aligned}
& \underset{u}{\text{minimize}}
& & E(u) =  \frac{\mu}{2} \inner{\nabla u, \nabla u} - \inner{u, \rho g}.\\
\end{aligned}
\end{equation}
\newcommand{\energygradient}{\frac{\delta E}{\delta u}}
We can validate this by computing the Euler-Lagrange equations:
\begin{align*}
    % \frac{\delta E}{\delta u} = \Part{\fancyL}{u} - \frac{d}{dx}\Part{\fancyL}{\nabla u}
    % notation?
    \frac{\delta E}{\delta u} = \Part{\fancyL}{u} - \frac{d}{dx}\Part{\fancyL}{u_x}
                              = -\rho g - \mu\Delta u = 0.
\end{align*}
We now introduce the incompressibility constraint $\nabla \cdot u = 0$, giving the constrained minimization
\begin{equation}\label{stokes_flow_optimization}
\begin{aligned}
& \underset{u}{\text{minimize}}
& & E(u) =  \frac{\mu}{2} \inner{\nabla u, \nabla u} - \inner{u, \rho g}\\
& \text{subject to}
& & \nabla\cdot u = 0.
\end{aligned}
\end{equation}
It is not immediately obvious how to form the constrained Euler-Lagrange equations here, as $\nabla\cdot$ is a differential operator.
We cannot just write
    $$\text{``}\frac{\delta E}{\delta u} = \lambda\nabla\cdot\text{''}$$
for scalar function $\lambda$, as we can with a pointwise linear constraint such as $u\cdot v = 0$ for some vector field $v$. However, this is just a problem of
notation. The evaluation of energy change with perturbations is defined as
\begin{align*}
    \inner{\frac{\delta E}{\delta u}, \delta u} = \int_\Omega \frac{\delta E}{\delta u}\cdot\delta u\,dx.
\end{align*}
We want this measure of energy change to be purely a divergence measure, up to a scalar multiplier $\lambda$:
\begin{equation}\label{el_pressure_constrained_div}
    \int_\Omega \frac{\delta E}{\delta u}\cdot\delta u\,dx = \int_\om \lambda \nabla\cdot \delta u\,dx.
\end{equation}
This means
that virtual displacements with $\nabla\cdot\delta u = 0$ will not cause an energy change, which is the condition that
we want for a stationary point.
We can now apply integration by parts to \eqref{el_pressure_constrained_div}, assuming that $\delta u$ vanishes on the boundary of the domain, to get
\begin{equation}
    \int_\Omega \frac{\delta E}{\delta u}\cdot\delta u\,dx = -\int_\om \nabla\lambda \cdot \delta u\,dx.
\end{equation}
We can now reasonably apply the localisation step to get the constrained Euler-Lagrange equations
\begin{equation}
\begin{split}
           \frac{\delta E}{\delta u} = -\nabla \lambda
    \quad\equiv\quad \mu\Delta u + \rho g - \nabla \lambda = 0.
\end{split}
\end{equation}
Along with the constraint $\nabla\cdot u = 0$, this is just the steady Stokes equations \eqref{steady_stokes}, where $\lambda = p$! We can see that the pressure $p$
is actually
a Lagrange multiplier, which measures a virtual force that responds to virtual displacements which would break the constraint of incompressibility.
In fact, we may think of this as a derivation of the pressure.

\subsubsection{Alternative direct derivation in terms of a modified energy}
Previously, we emphasized the meaning of the Lagrange multiplier. One utility of Lagrange's methods is their automated calculational power.
It is standard to express that a solution to the optimization
problem \eqref{stokes_flow_optimization}, with a differentiable equality constraint, is a stationary point of the modified energy
\begin{equation}\label{stokes_flow_modified_energy}
    L(u, \lambda) \coloneqq \frac{\mu}{2} \inner{\nabla u, \nabla u} - \inner{u, \rho g} - \inner{\lambda, \nabla\cdot u}.\\
\end{equation}
We can take an evaluated first variation with respect to $u$ to get
\begin{align*}
    \inner{\frac{\delta L}{\delta u}, \delta u} = \inner{-\rho g - \mu\Delta u, \delta u} - \inner{\lambda, \nabla\cdot\delta u},
\end{align*}
which by integration by parts becomes
\begin{equation}
    \inner{\frac{\delta L}{\delta u}, \delta u} = \inner{-\rho g - \mu\Delta u + \nabla \lambda, \delta u}.
\end{equation}
We then get
\begin{equation}
\begin{split}
    \frac{\delta L}{\delta u} &= -\rho g - \mu\Delta u + \nabla \lambda = 0,\\
    \frac{\delta L}{\delta \lambda} &= -\nabla\cdot u = 0,
\end{split}
\end{equation}
which are the steady Stokes equations \eqref{steady_stokes} with pressure $p = \lambda$.

% \subsection{Application to hydrostatics}
% For example, we may imagine the steady Stokes equations modelling a calm sea with a flat seabed.
% We can let the body force be gravity described by a potential $\phi$:
%     $$\rho g = -\nabla \phi.$$
% If we make a perturbed displacement of the velocity field at the bottom of the ocean, supposing that some volume of water
% is beginning to expand,
% we are working against gravity as well as our virtual force, pressure.
% 
% \vskip 0.2in
% (draw this)
% \vskip 0.2in
% >>>

\section{Scaling and dimension}
% <<<
\subsection{The Reynolds number}
% >>>


% \subsection{Kinds of fluids}
% \subsubsection{Incompressible fluids}
% \subsubsection{Inviscid flow}
% \subsubsection{Irrotational flow}
% \subsubsection{Steady flow}
% \subsubsection{Viscous flow and Newtonian fluids}

