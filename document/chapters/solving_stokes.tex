\section{Introduction}
% Introduction 
% <<<
The Stokes equations \eqref{unsteady_stokes}, which are solved for an incompressible Navier-Stokes flow,
assume the Reynolds number is $Re \ll 1$ and thus convective processes are neglible in comparison to viscous processes.
We will begin with the steady-state form \eqref{steady_stokes}.
Due to this simplification, the Steady stokes equations \eqref{steady_stokes}, repeated here:
\begin{align*}
    -\mu\Delta u + \nabla p = \rho g, \quad \nabla\cdot u = 0,
\end{align*}
form a constrained linear equation. As we saw in section \ref{pressure_derivation}, the pressure term $p$ is a Lagrange multiplier introduced
with the constraint $\nabla\cdot u = 0$.
\subsection{The lid-driven cavity flow problem}
A standard test case in computational fluid dynamics is the \textit{lid-driven cavity flow} problem.
We let the domain be is $\Omega = [-1,1]^2$ and define a Dirichlet boundary condition:
\begin{equation}\label{lid_driven_boundary_condition}
    \left.u\right|_\Gamma = u_\Gamma(x,y) =
    \left\{\begin{array}{lr}
        \left(1, 0\right)^T &\text{if $y = 1$,}\\
        \left(0, 0\right)^T &\text{otherwise}.\\
        \end{array}\right.
\end{equation}
We will use a finite element method to solve the steady Stokes flow \eqref{steady_stokes} with this domain and boundary condition.

% We will begin by discretizing the \textit{unconstrained} steady Stokes equations,
% which are a vector Poisson equation:
% \begin{equation}\label{steady_stokes_unconstrained}
%     -\mu\Delta u = \rho g.
% \end{equation}
% % >>>
% \subsection{Discretizing the vector Poisson equation}\label{discretizing_vector_poisson}
% % <<<
% In principle we should keep the Stokes equation
% in integral form (using the conservative-form Cauchy momentum equation \eqref{cauchy_continuity_eulerian}), and continue as we did
% in section \ref{trial_function}. However,
% we will take a formal step to skip the reasoning of section \ref{trial_function}, typical of finite element method derivations.
% As we start with the \textit{differential} equation \eqref{steady_stokes_unconstrained}, we can introduce a trial space $\Psi$ and then ``weaken''
% the equation by integrating against $v \in \Psi$, and removing the Laplacian by integration by parts:
% \begin{equation}\label{steady_stokes_unconstrained_weak}
%     \int_\Omega -\mu\Delta u\cdot v\,dx = \int_\Omega \rho g\cdot v\,dx
%     \quad\equiv\quad
%     \int_\Omega -\mu\nabla u : \nabla v\,dx = \int_\Omega \rho g\cdot v\,dx.
% \end{equation}

% Noting that the left-hand-side of \eqref{steady_stokes_unconstrained_weak} is a bilinear form in $u$ and $v$, and the right-hand-side
% is a linear functional in $\psi$, it is standard practice (ref) to write this kind of equation as
% \begin{equation}
%     a(u, v) = f(v).
% \end{equation}
% Our subsequent derivations are much the same as in \ref{discretizing_poisson}, simplified by our new notation.
% We can now approximate $u$ in the test space $\Phi$ as $\hat{u} = \sum_{i=1}^nu_i\phi_i$. By linearity we only need to compute
% trials over the basis trial functions $\psi_j$.
% We then have the linear system of equations
% \begin{equation}\label{elliptic_bilinear_form}
%     \sum_{i=1}^n u_i a\left(\phi_i, \psi_j\right) = f(\psi_j),\quad j=1,\cdots,n,
% \end{equation}
% which can be written in matrix form as
% \begin{equation}\label{elliptic_bilinear_form_matrix}
%     A\hat{u} = \begin{bmatrix}
%             a(\phi_1, \psi_1) & \cdots & a(\phi_1, \psi_n) \\
%             \vdots & & \vdots \\
%             a(\phi_n, \psi_1) & \cdots & a(\phi_n, \psi_n)
%             \end{bmatrix}
%     \begin{bmatrix} u_1 \\ u_2 \\ \vdots \\ u_{n-1} \\ u_n \end{bmatrix}
%     =
%     \begin{bmatrix} f(\psi_1) \\ f(\psi_2) \\ \vdots \\ f(\psi_{n-1}) \\ f(\psi_{n}) \end{bmatrix}
%     = \hat{f}.
% \end{equation}
% % The matrix $A$ is symmetric positive-definite, and we can therefore think of a solution to \eqref{elliptic_bilinear_form_matrix}
% % as as a minimizer of the scalar quadratic form
% % \begin{equation}\label{elliptic_quadratic_form}
% %     \hat{E}(\hat{u}) \coloneqq \frac{1}{2} \inner{\hat{u}, A\hat{u}} - \inner{\hat{u}, \hat{f}}.
% % \end{equation}
% % This is simply a discrete realisation of the fact that we can, as described in section \ref{pressure_derivation}, think
% % of a solution to the vector Poisson equation as a minimizer of the Dirichlet energy \eqref{steady_stokes_dirichlet_energy},
% % \begin{align*}
% %     E(u) \coloneqq \int_{\Omega} \frac{\mu}{2} \inner{\nabla u, \nabla u} - \rho g\cdot u \,dx.
% % \end{align*}
% We can solve \eqref{elliptic_bilinear_form_matrix} to get a velocity field $\sum_{i=1}^n u_i\phi_i$, although in general this will not satisfy $\nabla\cdot u = 0$.
% As some preliminary analysis, if $\Phi = \Psi$ and have the same basis functions, we have a symmetric-positive-definite system. This form of linear system is known to be stably solvable,
% for example by the conjugate gradient method.
% >>>


\subsection{The weak form of the steady Stokes equations}
% <<<
As described in section \ref{pressure_derivation}, the pressure $p$ is a Lagrange multiplier that appears
when solving the optimization problem \eqref{stokes_flow_optimization}:
\begin{equation*}
\begin{aligned}
& \underset{u}{\text{minimize}}
& & E(u) =  \frac{\mu}{2} \inner{\nabla u, \nabla u} - \inner{u, \rho g}\\
& \text{subject to}
& & \nabla\cdot u = 0.
\end{aligned}
\end{equation*}

\newcommand{\trialconstraint}{{\Psi_{\text{constraint}}}}
\newcommand{\testpressure}{{\Phi_{\text{pressure}}}}
As a first idea, we can introduce $p$ as a variable to solve for. Solving for the pressure (the ``dual variable'') simultaneously with the velocity
(the ``primal variable'')
is called a primal-dual method for the optimization \eqref{stokes_flow_optimization}, and the resulting finite element method is called \textit{mixed}.
Pressure then needs to be discretized, so we introduce another test space $\testpressure$.
To get a weak form of the steady Stokes equations \eqref{steady_stokes}, which are two equations including the constraint $\nabla\cdot u = 0$, we introduce
another trial space $\trialconstraint$, whose functions will be integrated against $\nabla\cdot u$. The weak form is then
\begin{equation*}
\begin{split}
    &\int_\om \left(-\mu\Delta u + \nabla p\right)\psi\,dx = \int_\Omega \rho g\cdot\psi\,dx,\\
    &\int_\om \left(\nabla\cdot u\right) q\,dx = 0, \quad\text{where $\psi \in \Psi, q \in \trialconstraint$},
\end{split}
\end{equation*}
which by integration by parts can be written as
\begin{equation}\label{steady_stokes_weak}
\begin{split}
    &\int_\om \mu\nabla u : \nabla \psi - p\nabla\cdot \psi\,dx = \int_\om \rho g\cdot \psi\,dx,\\
    &\int_\om -\left(\nabla\cdot u\right)q\,dx = 0, \quad\text{where $\psi \in \Psi, q \in \trialconstraint$}.
\end{split}
\end{equation}
% We will be using a Bubnov--Galerkin method, so $\Psi = \Phi$ and $\Psi_{\text{constraint}} = \Phi_{\text{constraint}}$.
To simplify notation, we can introduce the ``mixed spaces''
\begin{align*}
    \Phi_{\text{mixed}} = \Phi \times \Phi_{\text{constraint}}
    \text{\quad and \quad}
    \Psi_{\text{mixed}} = \Psi \times \Psi_{\text{constraint}}
\end{align*}
and, using the notation of \ref{taylor_hood_fenics}, \eqref{steady_stokes_weak} can be rewritten as
\begin{equation}\label{steady_stokes_bilinear_form}
    a((u, p), (\psi, q)) = L((v, q)), \quad\quad (u,p) \in \Phi_{\text{mixed}},\quad (\psi, q) \in \Psi_{\text{mixed}},
\end{equation}
where $a$ is a bilinear form defined by
\begin{align*}
    a((u, p), (\psi, q)) = \int_\Omega \nabla u : \nabla\psi - p\nabla\cdot \psi - \left(\nabla\cdot u\right)q\,dx
\end{align*}
and $f$ is a linear functional defined by
\begin{align*}
    L((v, q)) = \int_\Omega \rho g\cdot\psi\,dx.
\end{align*}

---todo: Use $H^1$, $L^2$, discretize spaces later.
\subsection{Discretizing the weak form}

In the derivations of chapter (), many of the manipulations were trivial applications of linearity and splitting of
boundary and interior terms, and is standard in the finite element literature to define forms such as the $a$ and $L$ above.
If we discretize to $n$ (interior) basis functions
there are $2n$ basis functions for the mixed test and trial spaces, which are the natural choice:
\begin{equation}
\begin{split}
    \Phi_{\text{mixed}} &= \text{span}\left(
        (\phi_1, 0),\cdots,(\phi_n, 0), (0,\phi_1^p),\cdots,(0,\phi_n^p)
    \right),\\
    \Psi_{\text{mixed}} &= \text{span}\left(
        (\psi_1, 0),\cdots,(\psi_n, 0), (0,\psi_1^p),\cdots,(0,\psi_n^p).
    \right)
\end{split}
\end{equation}

As in chapter (), we let $u$ be approximated by
    $$\tilde{u} = \tilde{u}_\Gamma + \tilde{u}_\text{interior} = \tilde{u}_\Gamma + \sum_{i=1}^n u_i\phi_i,$$
where $\tilde{u}_\Gamma \in \Phi^*$ approximates the Dirichlet boundary function $\left.u\right|_\Gamma = u_\Gamma$,
and $\tilde{u}_{\text{interior}} \in \Phi$ is the ``interior variation''. A boundary condition for pressure is not given
(---todo: Should the pressure have a Dirichlet boundary set to $0$?)
    $$\tilde{p} = \sum_{i=1}^n p_i\phi^p_i.$$


All we must do is plug these approximation definitions into the weak form \eqref{steady_stokes_bilinear_form}, resulting in the
$2n \times 2n$ linear system of equations:
\begin{equation}
\begin{split}
    a((\tilde{u}, \tilde{p}), (\psi_j, 0)) &= L((\psi_j, 0)), \quad j = 1,\cdots,n\\
    a((\tilde{u}, \tilde{p}), (0, \psi^p_j)) &= L((0, \psi^p_j)), \quad j = 1,\cdots,n.
\end{split}
\end{equation}

% The approximated pressure is $\tilde{p} \in \Phi_{\text{pressure}}$ where
%     $$\Phi_{\text{pressure}} = \text{span}(\phi_{pi},\cdots,\phi_{pn}).$$
% The pressure is zero on the boundary ($\left.p\right|_\Gamma = 0$), as it is only meaningful with respect to
% integrals over interior control volumes,
% so we have
%     $$\tilde{p} = \sum_{i=1}^n p_i \phi_{pi}.$$
% The velocity vector field $u$ is discretized as
%     $$\tilde{u} =  \phi_\Gamma + \phi = \phi_\Gamma + \sum_{i=1}^n u_i\phi_i$$
% where $\phi_\Gamma$ is the boundary approximation in $\Phi^*$, as in chapter ().

We simply plug these approximation definitions into the weak form \eqref{steady_stokes_weak}, resulting in the
linear system of equations:
\begin{equation}\label{steady_stokes_linear_system}
\begin{split}
&\sum_{i=1}^n \left[u_i\int_\Omega -\mu\nabla\phi_i : \nabla\psi_j\,d\hat{x}
        + p_i\int_{\Omega} -\psi_j \cdot \nabla\phi_{pi}\,d\hat{x}\right]
    = \int_\Omega \rho g\cdot\psi_j
        + \mu\nabla\phi_\Gamma :\nabla\psi_j \,d\hat{x},\\
&\sum_{i=1}^n u_i \int_\Omega -\phi_i \cdot \nabla q_j \,d\hat{x}
    = \int_\Omega \phi_\Gamma \cdot \nabla q_j \,d\hat{x},
\quad\quad j=1,\cdots,n.
\end{split}
\end{equation}
To simplify matters, we let the vector field test and trial spaces be direct sums of scalar function spaces:
$$
    \Phi^* = \Phi^*_s \oplus \Phi^*_s,
    \quad
    \Psi^* = \Psi^*_s \oplus \Psi^*_s,
$$
where the $s$ subscript denotes ``scalar''. This means
that the first equation in \eqref{steady_stokes_linear_system} splits into
two equations, for the $x$ and $y$ components of the velocity $u$.
Equation \eqref{steady_stokes_linear_system} is a $3n \times 3n$ linear system which (with the right choice of test and trial spaces) will be rank $3n - 1$,
as the pressure is defined up to a constant.

% Taylor-Hood elements
% https://fenicsproject.org/olddocs/dolfin/1.5.0/python/demo/documented/stokes-taylor-hood/python/documentation.html



% We introduce notation for the bilinear and linear forms in \eqref{steady_stokes_weak}:
% \begin{equation}
% \begin{split}
%     a(u, v) &\coloneqq \int_\om-\mu\nabla u : \nabla v\,dx,\quad\text{for $u \in \Phi, v \in \Psi$},\\
%     \hat{b}(p, v) &\coloneqq \int_\om-\left(\nabla\cdot v\right)p\,dx,\quad\text{for $p \in \testpressure, v \in \Psi$},\\
%     b(u, q) &\coloneqq \int_\om-\left(\nabla\cdot u\right)q\,dx,\quad\text{for $u \in \Phi, q \in \trialconstraint$},\\
%     f(v) &\coloneqq \int_\om \rho g\cdot v\,dx\quad\text{for $v \in \Psi$}.
% \end{split}
% \end{equation}
% Although they have the same form, $b$ and $\hat{b}$ are distinguished as they take inputs in different function spaces.
% We now have a simplified notation for the weak form \eqref{steady_stokes_weak},
% \begin{equation}\label{steady_stokes_weak_notation}
% \begin{split}
%     &a(u, v) + \hat{b}(p, v) = f(v),\\
%     &b(u, q) = 0, \quad\text{where $v \in \Psi, q \in \trialconstraint$}.
% \end{split}
% \end{equation}
% % Solving for $u^*$ in the equation $a(u^*, v) = f(v)$ is the standard vector Poisson equation, resulting in a symmetric-positive-definite
% % system when the test and trials spaces are discretized. We can imagine letting $p = 0$ and solving for $u^*$.
% % The first condition of \eqref{steady_stokes_weak_notation} will hold, but the second condition (does $b(u^*, q) = 0$?) generally will not.
% Working with discrete function spaces, we get a $2n\times 2n$ linear system in the unknowns $u_1,\cdots,u_n$ and $p_1,\cdots,p_n$,
% \begin{equation}
% \begin{split}
%     &\sum_{i=1}^n u_i a\left(\phi_i, \psi_j\right) + \sum_{i=1}^np_i\hat{b}\left(\phi^C_i, \psi_j\right) = f(\psi_j),\\
%     &\sum_{i=1}^nu_ib\left(\phi_i, \psi^C_j\right) = 0,\quad j=1,\cdots.n.
% \end{split}
% \end{equation}
% To emphasize the linear system structure of \eqref{steady_stokes_weak_notation}, the block matrix form is:
% \begin{equation}\label{steady_stokes_matrix}
% \def\arraystretch{1.5}
% \begin{split}
%     M\hat{x}
%     &= \begin{bmatrix}
%             A & \hat{B} \\
%             B & 0
%     \end{bmatrix}\hat{x} \\
%     &= \left[\begin{array}{@{}ccc|ccc@{}}
%             a(\phi_1, \psi_1) & \cdots & a(\phi_1, \psi_n)     & \hat{b}(\phi^C_1, \psi_1) & \cdots & \hat{b}(\phi^C_1, \psi_n) \\
%             \vdots & & \vdots                                  & \vdots & & \vdots \\
%             a(\phi_n, \psi_1) & \cdots & a(\phi_n, \psi_n)     & \hat{b}(\phi^Cn, \psi_1) & \cdots & \hat{b}(\phi^C_n, \psi_n) \\
%             \hline
%             b(\phi_1, \psi^C_1) & \cdots & b(\phi_1, \psi^C_n) & 0 &\cdots& 0      \\
%             \vdots & & \vdots \\                               & \vdots & & \vdots \\
%             b(\phi_n, \psi^C_1) & \cdots & b(\phi_n, \psi^C_n) & 0 &\cdots& 0       
%     \end{array}\right]
%     \left[\begin{array}{c} u_1 \\ \vdots \\ u_n \\ \hline p_1 \\ \vdots \\ p_n \end{array}\right]
%     =
%     \left[\begin{array}{c} f(\psi_1) \\ \vdots \\ f(\psi_{n}) \\ \hline 0 \\ \vdots \\ 0 \end{array}\right]
%     = \hat{b}.
% \end{split}
% \end{equation}
% \subsubsection{Is this method reasonable?}
% For the vector Poisson equation, letting $\Phi = \Psi$, we ended up with a symmetric-positive-definite system \eqref{elliptic_bilinear_form_matrix}, which is known to be stably solvable.
% We can ask how reasonable it is to solve \eqref{steady_stokes_matrix}, and what trial and test spaces we should use.
% In fact, in the problem \eqref{steady_stokes_weak_notation}, and more generally in a ``saddle point problem'', arising
% in Lagrange-multiplier methods for constrained PDEs, we should not choose just any test and trial spaces.
% The Ladyzhenskaya--Babu\v{s}ka--Brezzi condition, discussed later, enforces restrictions on choices that result in a stable method.
% We will until then continue with computations.
% 
% \subsubsection{Results and visualisation}
% \vskip 0.2in
% (results and visualisation)
% \vskip 0.2in
% % >>>
% \subsection{Discretizing the unsteady Stokes equations}\label{discretizing_unsteady_stokes}
% % <<<
% \newcommand{\uprev}{{u_{\text{prev}}}}
% The steady Stokes above are the stable state of the time-dependent Stokes flow,
% after the transient flow behaviour settles down. The unsteady Stokes equations \eqref{unsteady_stokes} are
% \begin{equation*}
%     \rho\Part{u}{t} = \mu\Delta u + \rho g - \nabla p, \quad \nabla\cdot u = 0.
% \end{equation*}
% These form an initial-boundary-value problem, and this will be our first attempt at discretizing a PDE in time.
% We could think of solving with the test and trial spaces over the domain $\Omega \times [0, T)$, but this is typically not done due to the memory costs,
% and different qualitative meaning of the time variable \cite{ham_fem}. Instead, we will use an implicit-Euler finite difference in time: 
% \begin{equation}\label{unsteady_stokes_implicit_euler}
%     \frac{\rho}{\Delta t} \left(u^{(n)} - u^{(n-1)}\right) = \mu\Delta u^{(n)} + \rho g - \nabla p^{(n)}, \quad \nabla\cdot u^{(n)} = 0,
% \end{equation}
% where $\Delta t$ is a fixed time step, and $u^{(n)}$ and $p^{(n)}$ is the solution at time $t_n = n\Delta t$. We can weaken each step
% \eqref{unsteady_stokes_implicit_euler}
% by integrating against trial functions $v \in \Psi$ and $q \in \trialconstraint$, performing integration by parts as in section \ref{discretizing_steady_stokes},
% and rearranging the knowns and unknowns. We can also let $u$ be $u^{(n)}$, $p$ be $p^{(n)}$, and $\uprev$ be $u^{(n-1)}$ in the above to simplify
% subsequent notation. The weak form of \eqref{unsteady_stokes_implicit_euler} is then:
% % \begin{equation}\label{unsteady_stokes_implicit_euler_weak}
% % \begin{split}
% %     \frac{\rho}{\Delta t} \int_\om \left(u^{(n)} - u^{(n-1)}\right)\cdot v\,dx
% %         &= \int_\om \mu\nabla u^{(n)}:\nabla v + \rho g\cdot v + \left(\nabla\cdot v\right) p^{(n)}\,dx,\\
% %     \quad \int_\om \left(\nabla\cdot u^{(n)}\right) q\,dx &= 0.
% % \end{split}
% % \end{equation}
% \begin{equation}\label{unsteady_stokes_implicit_euler_weak}
% \begin{split}
%     \int_\om \frac{\rho}{\Delta t} u\cdot v - \mu\nabla u:\nabla v - \left(\nabla\cdot v\right)p\,dx
%         &= \int_\om \frac{\rho}{\Delta t}\uprev\cdot v + \rho g \cdot v\,dx,\\
%     \quad \int_\om \left(\nabla\cdot u\right) q\,dx &= 0,
% \end{split}
% \end{equation}
% % and with the linear form definitions in section \ref{discretizing_steady_stokes} this is
% % \begin{equation}\label{unsteady_stokes_implicit_euler_weak_notation}
% % \begin{split}
% %     &\int_\om \frac{\rho}{\Delta t} u\cdot v \,dx + a(u, v) + \hat{b}(p, v) = \int_\om \frac{\rho}{\Delta t} \uprev\cdot v \,dx + f(v),\\
% %     &b(u, q) = 0, \quad\text{where $v \in \Psi, q \in \trialconstraint$}.
% % \end{split}
% % \end{equation}
% We can define the linear forms as
% \begin{equation}
% \begin{split}
%     a(u, v) &\coloneqq \int_\om\frac{\rho}{\Delta t}u\cdot v -\mu\nabla u : \nabla v\,dx,\quad\text{for $u \in \Phi, v \in \Psi$},\\
%     \hat{b}(p, v) &\coloneqq \int_\om-\left(\nabla\cdot v\right)p\,dx,\quad\text{for $p \in \testpressure, v \in \Psi$},\\
%     b(u, q) &\coloneqq \int_\om-\left(\nabla\cdot u\right)q\,dx,\quad\text{for $u \in \Phi, q \in \trialconstraint$},\\
%     f(v) &\coloneqq \int_\om \frac{\rho}{\Delta t}\uprev\cdot v + g\cdot v\,dx\quad\text{for $v \in \Psi$}
% \end{split}
% \end{equation}
% to reexpress \eqref{unsteady_stokes_implicit_euler_weak} in the notation
% \begin{equation}
% \begin{split}
%     &a(u, v) + \hat{b}(p, v) = f(v),\\
%     &b(u, q) = 0, \quad\text{where $v \in \Psi, q \in \trialconstraint$}.
% \end{split}
% \end{equation}
% This is the same structure as in the steady Stokes system \eqref{steady_stokes_weak_notation},
% and so the matrix block structure is the same as in \eqref{steady_stokes_matrix}. Therefore, every step we need to solve a linear system
% that is very similar to the steady Stokes problem. In fact this step can be thought of as successively introducing the momentum source $\rho g$
% (ignoring convection), while solving for the updated pressure needed to keep the fluid non-compressed.
% 
% \subsubsection{Discretizing the initial condition}
% We may have some analytically determined, or otherwise, initial velocity field $u$ with $\nabla\cdot u = 0$.
% We would like to form $u^{(0)}$ in order to start the iteration. The velocity $u$ should be projected in some way into the test space $\Psi$.
% Enforcing $u^{(0)}$ to give the same ``blurred average'' value when evaluated against a trial function,
% \begin{equation}\label{initial_velocity_projection}
%     \int_\om u^{(0)}\cdot v\,dx = \int_\om u \cdot v\,dx,\quad \forall v \in \Psi,
% \end{equation}
% gives a linear system
% \begin{equation}\label{initial_velocity_projection_linear_system}
%     \sum_{i=1}^n u^{(0)}_i \int_\om \phi_i \cdot \psi_j\,dx = \int_\om u\cdot \psi_j\,dx,
%     \quad j=1,\cdots,n.
% \end{equation}
% Solving this linear system for the $u^{(0)}_i$ gives $u^{(0)} = \sum_{i=1}^n u^{(0)}_i \phi_i$ as a projection of $u$ onto $\Phi$.
% This projection is orthogonal if $\Phi = \Psi$, and therefore could be considered the ``best'' such projection in the Euclidean norm.
% This is the standard Gramian matrix construction for projection in approximation theory \cite{approximation_theory}.
% 
% % >>>
% >>>
% 
% \section{Solving non-linear equations}
% % <<<
% \subsection{A non-linear Poisson equation}
% \subsection{A non-linear heat equation}
% \subsection{The Burgers equation}
% % >>>
% 
% \section{Implementing finite element methods}
% % <<<
% \subsection{The Ciarlet definition of a finite element space}
% \cite{ciarlet}, \cite{ham_fem}, \cite{fenics_book}
% 
% 
% % Discuss Ciarlet definition, difference between FEM, spectral etc. FEM uses a domain partition to help in constructing the basis functions.
% 
% One great utility of the finite element method is that it is compatible with complex geometric domains and domain partitions.
% Some greater effort is needed to apply finite differences correctly across complex boundaries, and it is non-trivial to implement
% a varying resolution of the discretisation. In the finite element method, however, specifying the test and trial functions over
% a square grid is much the same as specifying them over, for example, a surface mesh of arbitrary topology. For modelling, for example,
% heat transfer in a complex solid, one can construct basis functions over a grid on the interior, and over cut-off boundary cells near the exterior.
% This is all in principle, of course, as one needs to
% \begin{enumerate}
%     \item Break the domain up into small pieces.
%     \item Construct the basis and trial functions over this domain partition (e.g. by finding polynomial coefficients).
%     \item Compute all inner products of test and trial functions, and their relevant derivatives, either numerically or analytically.
%     \item Solve possibly many huge sparse linear systems, while possibly changing the structure of the domain partition (requiring changes to the inner products).
% \end{enumerate}
% 
% Step (1) is already a field in itself, as evidenced by the open source tool TetGen \cite{tetgen}.
% TetGen is a small tool whose primary purpose is to perform \textit{constrained Delaunay tetrahedralization}, given a solid boundary and a point cloud on the interior.
% This constructs a valid tetrahedral partition intended for finite element solvers. The partition is efficiently created in a very robust manner in over 36k lines of C code. TetGen is part of the geometric backbone of many FEA tools (references).
% % >>>
