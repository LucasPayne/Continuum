
\chapterprecishere{
Corpus omne perseverare in statu suo quiescendi vel movendi uniformiter in directum, nisi quatenus a viribus impressis cogitur statum suum mutare.
\\\\
Mutationem motus proportionalem esse vi motrici impressae, \& fieri secundum lineam rectam qua vis illa imprimitur.
\\\\
Actioni contrariam semper \& aequalem esse reactionem: sive corporum duorum actiones in se mutuo semper esse aequales \& in partes contrarias dirigi.
\par\raggedleft \textup{Newton \cite{newton}}}
% \section{Newton's laws of motion} % <<<
% 
% Newton's three laws, namely those of inertia, force, and equilibrium, have found universal success in application
% to mechanical systems such as the pendulum, the motion of a rigid body, the evolution of a bending beam, and, as we shall see,
% the motion of a fluid. \textit{Mechanics} could be thought of as the study of physical motion, but the word ``physical'' might be misleading.
% Newton's principles are mathematical in nature, applicable to the study of motion in a general sense as some unambiguously
% measurable state which evolves in time.
% 
% \subsection{Symmetry, momenta, and inertia}
% Mechanics as a theory of physical motion will require a definition of physical motion. A first attempt might be to posit
% that ``physical states'' are representable as points in a finite-dimensional manifold, which we call the configuration space $C$, which is the case for typical
% notions of state such as the two angles in a double pendulum, or the position and orientation of a rigid body. We might define a motion as
% a continuous time-parameterised curve
%     $$\gamma: [t_1, t_2] \rightarrow C.$$
% 
% % (--- motivation of $F = ma$, and momenta as fundamental quantity).
% 
% We start in the middle:
% \begin{equation}
%     \text{Total force = change of momentum}.
% \end{equation}
% In this form, Newton's second law of motion states that a (non-explanatory) measurement of change of momentum will be called ``force''.
% 
% 
% 
% % >>>
% \section{The Euler-Lagrange equations: \small{From $F = ma$ to $\Part{\fancyL}{q} - \frac{d}{dt}\Part{\fancyL}{\dot{q}} = 0$}} % <<<
% \subsection{A Lagrangian of a mechanical system}
% Force is an intensive measurement of the change in momentum. In the language of calculus,
% \begin{equation}\label{force_equation}
%     \int_{s_1}^{s_2} F\,dt = \fancyP(s_2) - \fancyP(s_1)
% \end{equation}
% for all time subintervals $[s_1, s_2] \subset [t_1, t_2]$. If we restrict $F$ to be conservative and a function only of position $q$, then we may
% let $F = -\Part{V}{q}$ for some potential function $V$. Suppose also that $\mathcal{P} = \Part{T}{\dot{q}}$ for some potential function
% (called the ``kinetic energy'') independent
% of position $q$. We then define a Lagrangian of the mechanical system to be
%     $$\fancyL(q, \dot{q}, t) = T(\dot{q}, t) - V(q, t) = \text{kinetic} - \text{potential}.$$
% By definition, we have the force equation \eqref{force_equation} as
%     $$-\int_{s_1}^{s_2} \Part{\fancyL}{q}\,dt = \Part{\fancyL}{\dot{q}}(s_2) - \Part{\fancyL}{\dot{q}}(s_1).$$
% The step toward the calculus of variations (---)
% \begin{equation}\label{el_inner_product_sum}
%         \int_{s_1}^{s_2} \Part{\fancyL}{q}h\,dt + \int_{s_1}^{s_2} \Part{\fancyL}{\dot{q}}\frac{dh}{dt}\,dt = 0
%     \quad\equiv\quad    \left<\Part{\fancyL}{q}, h\right> + \left<\Part{\fancyL}{\dot{q}}, \frac{dh}{dt}\right> = 0.
% \end{equation}
% Adjointness of differential operator $\frac{d}{dt}$ to $-\frac{d}{dt}$, and integration by parts. (---)
% By linearity, we then get the reformulation of \eqref{el_inner_product_sum} as
% \newcommand{\gateauxlagrangian}{\Part{\fancyL}{q} - \frac{d}{dt}\Part{\fancyL}{\dot{q}}}
%     $$\left<\gateauxlagrangian, h\right> = 0$$
% for all perturbation functions $h$.
% 
% \subsection{The first variation of a functional}
% In the calculus of variations, $\gateauxlagrangian$ is an instance of the \textit{G\^ateaux derivative}, also called the ``first variation'' of
% the functional
% \begin{align*}
%     S[q] \coloneqq \int_{t_1}^{t_2} \fancyL(q, \dot{q}, t)\,dt.
% \end{align*}
% The first variation measures response of the value of $S$, called the \textit{action}, to perturbations of the (differentiable) input function $q$,
% and is denoted
% \begin{equation}
%     \frac{\delta S}{\delta q(t)} \coloneqq \gateauxlagrangian.
% \end{equation}
% The first variation is linear in the perturbation function, and so another term for the G\^ateaux derivative could be ``functional gradient''.
% Setting this to zero gives the Euler-Lagrange equations, and the practice of determining trajectories of motions as stationary curves of the action is called the ``principle of stationary action''.
% 
% \subsection{From the Lagrangian to the equations of motion}
% In the framework of Lagrangian mechanics, $\fancyP \coloneqq \Part{\fancyL}{\dot{q}}$ is the momentum. If there are $d$ degrees of freedom
% in the mechanical system, and we suppose that $q_1,\cdots,q_d$ are (local) variables of state, then we say that $\fancyP_i \coloneqq \Part{\fancyL}{\dot{q}_i}$
% are \textit{conjugate} to the $q_i$.
% 
% (---) By inducing the equations of motion by a Lagrangian, we get systems with a ``physical interpretation'' with ``physically meaningful'' force measurements.
% 
% \subsection{Why ---?}
% These variational ideas appear because, no matter if our configuration space is finite-dimensional, time forms a continuum, and therefore if we
% globally consider the calculus of the motion as a whole, it must be a ``variational'' calculus. (---)
% 
% (---) Physical process, conservation laws, conservative forces (hint at thermodynamics limiting the range of stress tensors?)
% 
% When we consider mechanical models of continuum processes, we will see that these same ideas appear in the spatial dimensions too.
% A variational understanding of a continuum mechanics model leads very easily to a class of methods called Galerkin methods for solving
% the corresponding (PDE) equations of motion.
% 
% 
% % >>>

The calculus of variations.
