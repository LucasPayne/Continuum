% Terry Tao, on writing: https://terrytao.wordpress.com/advice-on-writing-papers/
% https://terrytao.wordpress.com/2007/03/18/why-global-regularity-for-navier-stokes-is-hard/#comment-625243
% \documentclass{article}
\documentclass[11pt,a4paper]{memoir}
% \setsecnumdepth{subsubsection}
\setsecnumdepth{subsection}
\setlrmarginsandblock{3cm}{3cm}{*} % Centre adjustment
\setulmarginsandblock{2.5cm}{*}{1}
\checkandfixthelayout 

\usepackage{amsmath}
\usepackage{amssymb}
\usepackage{mathtools}
\usepackage{microtype}
\usepackage{dirtytalk}
\usepackage{relsize}
\newcommand{\bb}{\begin{bmatrix}}
\newcommand{\bbe}{\end{bmatrix}}
\newcommand{\pr}{\prime}
\newcommand{\ppr}{{\prime\prime}}
\newcommand{\pppr}{{\prime\prime\prime}}
\newcommand{\inner}[1]{\left<#1\right>}
\newcommand{\fancyA}{\mathcal{A}}
\newcommand{\fancyL}{\mathcal{L}}
\newcommand{\fancyN}{\mathcal{N}}
\newcommand{\norm}[1]{\left\Vert#1\right\Vert}
\newcommand{\om}{\Omega}
\newcommand{\pom}{{\partial\Omega}}
\newcommand{\diver}{\text{div}}
\newcommand{\Part}[2]{\frac{\partial #1}{\partial #2}}
% footnotes
\renewcommand\footnoterule{}

\newcommand{\todo}[1]{\vskip 0.1in \hrule \vskip 0.03in {#1} \vskip 0.03in \hrule \vskip 0.1in}

\title{Variational methods for the Navier-Stokes equations
\scriptsize{(working title)}
}
\author{Lucas Payne}

\begin{document}
\maketitle

\tableofcontents

\chapter{Continuum mechanics}
\section{On $F = ma$} % <<<
\begin{equation}
    \text{Total force = change of momentum}.
\end{equation}
In this form, Newton's second law of motion states that there is such a thing called ``force'',
a non-explanatory measurement made along a trajectory of a mechanical system, which is used to measure change in momentum.
Newton's three laws, namely those of inertia, force, and equilibrium, have found universal success in application
to mechanical systems. Separate from any particular notion of position and space, the principles of Newtonian mechanics
apply to complex systems, and as we shall see, are fundamental to the derivation of equations of fluid motion.
A primary example is that of a free rigid body, say, a cube of constant density. Under the assumption of rigidity,
the state of the rigid body can be encoded with six numbers. These numbers describe position and orientation, which are paired
to \textit{linear momentum} and \textit{angular momentum}.

\subsection{The Newtonian free particle}
We suppose that we have some space, called a configuration space, describing the possible states of the mechanical system. We suppose here
that this space forms a finite dimensional manifold, which is true for the typical unconstrained configuration spaces such as those of rigid
body motions.

\subsubsection{Symmetry, momenta, and inertia}

Consider first the idea of a free particle in $X = \mathbb{R}^3$ describing a traditional point in space that can move by translation.
One fundamental assumption is that the laws of physics should ``look the same'' from the point of view of each point.
This is formalized by a group action, here
    $$g \in G \cong (\mathbb{R}, +),\quad x\in X,\quad g\cdot x = x + g.$$
Of course, $G$'s action has only one orbit, as in, it is transitive. Denoting by $G\cdot x$ the orbit of $x$ under the action of $G$,
the group action induces an equivalence relation on $X$:
    $$x, y \in X,\quad x \sim y \Leftrightarrow G\cdot x = G\cdot y.$$
Since there is only one orbit, we see that $x \sim y$ for all $x, y \in X$, as in, we are considering each point as ``the same''.
% In this case, $G$ forms a topological (in particular, Lie)  group which is homeomorphic to the space that is acting on. If we may consider mechanics
% as the study of motion, then it seems suitable that our kinematics, the study of trajectories of motion in the purely geometrical sense,
% should study curves in this topological group. We may distinguish an ``initial point/state'', which we normalize to the group identity.
% In this conception, we consider ``curves of translations'' rather than curves of points/states.
%
% We, of course, have ambiguity in the choice of point to normalize to the group identity. If we have two curves
%     $$\gamma_1 : [0,1] \rightarrow X,\quad \gamma_2 : [0,1] \rightarrow X,$$
% Suppose there was a canonical idea of a body ``at rest''. Now, suppose a body exhibits some motion.

\subsubsection{Energy and time}
\subsection{The Lagrangian free particle}
\subsubsection{The kinetic energy}
\subsubsection{From $F=ma$ to the principle of least action}

\subsection{Systems of forces}
Force has been defined as a non-explanatory measurement. Here, discuss physical modelling with prescribed forces.
\subsubsection{The Euler-Lagrange equations}


% >>>
\section{From $F = ma$ to $\Part{\fancyL}{q} - \frac{d}{dt}\Part{\fancyL}{\dot{q}} = 0$: The Euler-Lagrange equations} % <<<
% >>>
\section{From $F = ma$ to $\frac{Du}{Dt} = \frac{1}{\rho}\nabla\cdot\sigma + f$: The Cauchy momentum equations} % <<<
% >>>
\section{The Cauchy momentum equations as Euler-Lagrange equations} % <<<
% >>>
\section{Transport of quantity and continuity equations} % <<<
% >>>
\section{The continuum hypothesis and constitutive relations} % <<<
% >>>

\chapter{The Navier-Stokes equations}
% \section{The continuum hypothesis}
% \section{Body and stress forces}
% \section{Conservation laws}
% \subsection{Conservation of mass}
% \subsection{Conservation of linear momentum}
% \subsection{Conservation of angular momentum}
% \subsection{Conservation of energy}
% \subsection{Differentialization and coordinates}
% Navier-Stokes in expanded differential form for Cartesian coordinates. Maybe useful for finite differences.
% Working backwards to the weak form doesn't make much sense.
% \subsection{Kinds of fluids}
% \subsubsection{Incompressible fluids}
% \subsubsection{Inviscid flow}
% \subsubsection{Irrotational flow}
% \subsubsection{Steady flow}
% \subsubsection{Viscous flow and Newtonian fluids}

\chapter{The finite element method}
\section{Discretizing variational principles}
% The weak form is exactly analagous to perturbation principle in the principle of least action.
% To actually be an instance of the principle of least action, we need a Lagrangian. For example, the Poisson
% equation is the Euler-Lagrange equation of the Dirichlet energy.


\begin{thebibliography}{9}
\bibitem{johann_bernoulli}
Johann Bernoulli, \textit{``Problema novum ad cujus solutionem Mathematici invitantur.'' (A new problem to whose solution mathematicians are invited.)}, 1696.
(retrieved from wikipedia/brachistochrone\_curve)

\bibitem{dirichlet_principle}
A. F. Monna, \textit{Dirichlet's principle: A mathematical comedy of errors and its influence on the development of analysis}, 1975

\bibitem{pde_larsson}
Stig Larsson, \textit{Partial differential equations with numerical methods}, 2003

\bibitem{lax_1973}
Peter Lax, \textit{Hyperbolic Systems of Conservation Laws and the Mathematical Theory of Shock Waves}, 1973

\bibitem{lanczos}
Cornelius Lanczos, \textit{The Variational Principles of Mechanics}, 1952

\bibitem{batchelor}
G. K. Batchelor, \textit{Introduction to Fluid Dynamics}, 1967

\bibitem{leal}
L. Gary Leal, \textit{Advanced Transport Phenomena: Fluid Mechanics and Convectice Transport Processes}, 2007

\bibitem{fem_ns}
Vivette Girault, Pierre-Arnaud Raviart, \textit{Finite Element Methods for Navier-Stokes equations}, 1986


\end{thebibliography}

\end{document}
