% Terry Tao, on writing: https://terrytao.wordpress.com/advice-on-writing-papers/
% https://terrytao.wordpress.com/2007/03/18/why-global-regularity-for-navier-stokes-is-hard/#comment-625243
\documentclass{article}
\usepackage{amsmath}
\usepackage{amssymb}
\usepackage{mathtools}
\usepackage{microtype}
\usepackage{dirtytalk}
\newcommand{\bb}{\begin{bmatrix}}
\newcommand{\bbe}{\end{bmatrix}}
\newcommand{\pr}{\prime}
\newcommand{\ppr}{{\prime\prime}}
\newcommand{\pppr}{{\prime\prime\prime}}
\newcommand{\inner}[1]{\left<#1\right>}
\newcommand{\fancyA}{\mathcal{A}}
\newcommand{\fancyL}{\mathcal{L}}
\newcommand{\fancyN}{\mathcal{N}}
\newcommand{\norm}[1]{\left\Vert#1\right\Vert}
\newcommand{\om}{\Omega}
\newcommand{\pom}{{\partial\Omega}}
\newcommand{\diver}{\text{div}}
\newcommand{\pdf}[2]{\frac{\partial #1}{\partial #2}}
% footnotes
\renewcommand\footnoterule{}

\newcommand{\todo}[1]{\vskip 0.1in \hrule \vskip 0.03in {#1} \vskip 0.03in \hrule \vskip 0.1in}

\title{A Treatise on the Finite Element Method
\scriptsize{(working title)}
}
\author{Lucas Payne}

\begin{document}
\maketitle

% Physical derivation of the heat equation
% Fick's law, continuity equation
% Conservation law
% Constitutive relations
% 
% Continuum mechanics
%     Cauchy stress tensor
%     The principle of linear momentum
% Deviatoric stresses
% The Reynolds transport theorem
% 
% Weak solutions to partial differential equations
% The Euler-Lagrange equations


\begin{itemize}
\item The analytic theory of variational methods.
      Euler-Lagrange equations.
\item Try and derive and describe integration by parts intuitively (Why should it be possible? What does it do to differentiability
requirements and why is this allowable, what does this imply?)
\item Discussion of Dirichlet's principle and resolution of problems with Sobolev spaces, distributions, etc.
\item Examples. Poisson's problem analysis.

\item The finite element method, overview and application to Poisson's problem (linear), Burger's equation (non-linear).
\item Galerkin methods.

\item ...
\item Using the finite element method to solve some very difficult problem (such as the Navier-Stokes equations in 2D).
\item ...
\item (possibly appendices)
\item Numerics, discussion of libraries for equation solving (petsc, linpack, lapack),
standard numerical tools used often in finite element solving. Possibly dicuss issues
such as conditioning and precision.


\end{itemize}
\newpage


\tableofcontents
\newpage

\section{Introduction}
\todo{
Modelling physical processes and problems with partial differential equations, then solving them numerically.
Possibly a physically-motivated derivation of the heat equation as described in Larsson.

}

\section{Variational calculus}
\subsection{Some history and motivation}

The variational calculus is effectively the calculus of function spaces. Variational methods consider the local changes of functionals with respect to perturbations
of their input functions. The calculus stems from techniques developed around the turn of the 18th century. A classical problem of a variational nature,
which was a great motivator for the development of these methods, is a challenge posed by Swiss mathematician Johann Bernoulli.
The challenge was taken up in 1697 by Sir Isaac Newton, Gottfried Leibniz, Marquis de l'H\^opital, and Jacob Bernoulli, among others. The
brachistochrone problem, as stated by Bernoulli, is \cite{johann_bernoulli}

\say{
Given two points A and B in a vertical plane, what is the curve traced out by a point acted on only by gravity, which starts at A and reaches B in the shortest time.
}

A fundamental idea is that the optimal curve (a function $C:[0,1]\rightarrow \mathbb{R}^2$) is the minimizer of a certain ``energy'' functional $E$,
a function from the space of feasible curves to $\mathbb{R}$. A \textit{functional derivative} is defined, leading to a theorem in complete analogy to the principle, attributed to Fermat, that a necessary condition for $x^*$ to be a minimizer of a differentiable function $f(x)$ is that the function is stationary at $x^*$:
    $$\nabla f(x^*) = 0.$$
One such derivative is the \textit{G\^ateaux derivative}, denoted
$\frac{dE}{du}$, which will be defined below.
$u$, in a certain sense, is from the space of possible infinitesmal deformations of the curve, which preserve required boundary conditions and differentiability. With a functional derivative defined, the analogue to Fermat's theorem is that
    $$\frac{dE}{du}(u^*) = 0$$
is necessary for $u^*$ to be a minimizer. This PDE is called the \textit{Euler-Lagrange equation} of the minimization problem, and, in particular, gives the
exact minimizer if the minimized energy $f(u)$ is strictly convex.
\footnote{
$f(u)$ is strictly convex if $\theta f(u) + (1-\theta)f(v) > f(\theta u + (1 - \theta) v)$ for $0 < \theta < 1$. For $f \in C^2$ this means that the function has strictly positive curvature, and thus no non-global minimizers.
}
\todo{Application to brachistochrone problem, deriving $\cosh$ solution by defining a suitable energy}


A more fundamental example is given by Poisson's equation,
\begin{equation}\label{Poisson}
    \Delta u = g,
\end{equation}
which is the Euler-Lagrange equation of a certain quadratic (and therefore convex) energy minimization problem, minimizing what is called the \textit{Dirichlet energy},
$f(u) = \int_\Omega \norm{\nabla u}^2\,dx.$
This idea, and its validity, was heavily investigated by mathematicians [---when and by who?], and is called \textit{Dirichlet's principle} \cite{dirichlet_principle}. The minimizer of $f$ is not necessarily a solution to \eqref{Poisson}, but it is necessarily a solution to what is called
the \textit{weak} form of \eqref{Poisson},
\begin{equation}\label{WeakPoisson}
    \int_\Omega u \nabla \phi\,dx  = \int_\Omega g\phi\,dx.
\end{equation}
(---dubious)
\todo{Correct details}


\subsection{Deriving the Euler-Lagrange equations} % <<<
The functional derivative and the Euler-Lagrange equations are the very basic tools of variational calculus,
directly analagous to the gradient and Fermat's theorem of stationary points. Here they are derived for a certain
class of functionals, those defined by density integrals. Some examples of functionals of this kind are:
\begin{itemize}
\item
    The Dirichlet energy, $E(u) = \int_\Omega \norm{\nabla u}^2\,dx$, which is the Euclidean square-norm of the gradient.
    Minimization leads to harmonic functions, solutions of $\nabla u = 0$ on the interior.
\item
    $E(u) = \int_\Omega \frac{1}{2}\norm{\nabla u}^2 + ug\,dx$ for some $g:\Omega\rightarrow\mathbb{R}$ defined on the interior of $\Omega$.
    Minimization leads to solutions of Poisson's equation, $\nabla u = g$ on the interior or exterior of the closed domain. This is used
    for example to compute gravitational potential fields.
\item
    The length of a curve segment, where $c:[0,1]\rightarrow \mathbb{R}^2$ is differentiable, $E(c) = \int_0^1 \norm{c^\prime(x)}\,dx$.
\item
    The area of a surface patch, where $s:[0,1]^2\rightarrow \mathbb{R}^3$ is differentiable, $E(s) = \int_0^1\int_0^1 \sqrt{\det(J_s(x,y)^T J_s(x,y))}\,dx\,dy$.
    Minimizing this energy is a case of what is called Plateau's problem.
\item
    $E(u) = \int_\Omega \norm{\nabla u} + \frac{\lambda}{2} \norm{f(x) - u(x)}^2\,dx,$
    an energy whose minimization balances the total variation and the deviation from some target function.
    This is a popular functional in image denoising.
\item
    (todo: correct physics) The work done by a particle moving as $c:[0,1]\rightarrow \mathbb{R}$ in force field $v$,
    $\int_0^1 c^\prime(t)\cdot v(c(t))\,dt$.
\item
    A light ray travels in a curve $c:[0,1]\rightarrow \mathbb{R}^2$ with endpoints predetermined, through an inhomogeneous medium with varying speed-of-light $\phi(x) > 0$.
    Then $E(c) = \int_0^1 \frac{c^\prime(t)}{\phi(c(x))}\,dx$ is minimized by Fermat's principle of least time.
\end{itemize}
The discussion here is not rigorous.

\subsubsection{The G\^ateaux derivative}
\todo{Define $U$ to be a suitably general function space for which this works.}

Suppose we have some functional energy $E$, a function of $u \in U$ and its first $n$ derivatives,
which is a density integral over the \textit{Lagrangian} $\fancyL$,
    $$E = E(u, u^\prime, \cdots, u^{(n)}) = \int_\Omega \fancyL(u(x),u^\prime(x),\cdots,u^{(n)}(x))\,dx.$$
Here $u^{(k)}$ is understood to be the tensor of all $k$'th partial derivatives. To measure
the response of the functional $E$ to small perturbations of $u$, $u + \epsilon h$ for some $h$, we need $u + \epsilon h$ to
have sufficiently many derivatives and bounded integrals (or is ``admissable'' in Dirichlet's terminology \cite{dirichlet_principle}). Suppose $h$ is defined to be of some function space $V$ such that
this is true.
The G\^ateaux derivative is derived through taking limits in the exact same way as the regular derivative.
If $u$ is perturbed by $h$ then $u^\prime$ is perturbed by $h^\prime$, and so on. If the limit exists, then the
linear response of $E$ to this perturbation is
\begin{align*}
    \frac{dE}{du}\Big|_h &\coloneqq \lim_{\epsilon \rightarrow 0} \frac{1}{\epsilon} \left(E(u + \epsilon h, u^\prime + \epsilon h^\pr, \cdots, u^{(n)} + \epsilon h^{(n)})
        - E(u, u^\pr, \cdots, u^{(n)})\right).\\
\end{align*}
For energy functionals based on a Lagrangian $\fancyL$, this gives by Taylor expansion of $E$ with respect to the $u^{(i)}$,
\begin{equation}\label{gateaux_lagrangian}
\begin{split}
    \frac{dE}{du}\Big|_h &=
        \lim_{\epsilon \rightarrow 0} \frac{1}{\epsilon} \int_\Omega \mathcal{L}(u +\epsilon h, u^\pr + \epsilon h^\pr,\cdots, u^{(n)} + \epsilon h^{(n)})
            - \mathcal{L}(u, u^\pr,\cdots, u^{(n)})\, dx \\
        &= \lim_{\epsilon \rightarrow 0} \frac{1}{\epsilon}
                \int_\Omega \mathcal{L}
                    + \epsilon h\frac{\partial\mathcal{L}}{\partial u}
                    + \epsilon h^\pr\frac{\partial\mathcal{L}}{\partial u^\pr}
                    +\cdots
                    + \epsilon h^{(n)}\frac{\partial\mathcal{L}}{\partial u^{(n)}} + O(\epsilon^2)
                    - \mathcal{L}\,dx\\
        &= \int_\Omega h\frac{\partial\mathcal{L}}{\partial u} + h^\prime\frac{\partial\fancyL}{\partial u^\prime} + \cdots + h^{(n)}\frac{\partial\fancyL}{\partial u^{(n)}}\,dx
\end{split}
\end{equation}
\todo{Explain why this result is expected, as it just adds up the responses of the density.}

This is a suitable definition of a ``functional derivative'' of an energy defined by a Lagrangian,
but it would be nice to form instead a ``functional gradient'', an object encoding functional derivative information
in all directions of perturbation.
In analogy to the gradient encoding all directional derivatives,
\begin{align*}
    \lim_{\epsilon\rightarrow 0} \frac{f(x + \epsilon d) - f(x)}{\epsilon} = \inner{\nabla f, d},
\end{align*}
the functional gradient $\frac{dE}{du}$ is defined such that
    $$\frac{dE}{du}\Big|_h
    = \int_\Omega h\frac{\partial\mathcal{L}}{\partial u} + h^\prime\frac{\partial\fancyL}{\partial u^\prime} + \cdots + h^{(n)}\frac{\partial\fancyL}{\partial u^{(n)}}\,dx
     = \inner{\frac{dE}{du}, h} = \int_\Omega \frac{dE}{du}(x)h(x)\,dx.
$$
This is possible due to what at first may seem like an unintuitive trick, repeated integration by parts.



\subsubsection{Integration by parts}
\todo{Make good derivation and reasoning for integration by parts. Is exterior calculus necessary? It seems
to be for making sense of multi-dimensional integration by parts. Introduce any formalisms used in the derivation.}
\textit{Integration by parts} is a generalization of the generalized Stokes theorem.
The \textit{generalized Stokes theorem} is written
\begin{equation}\label{generalized_stokes}
    \int_\om d\omega = \int_\pom \omega.
\end{equation}

Integration by parts formulae are a consequence of the (alternating) Leibniz product rule satisfied by the exterior derivative,
\begin{align*}
    \int_\om d(\phi\wedge \omega) = \int_\pom \phi\wedge\omega
        = \int_\om d\phi \wedge \omega + (-1)^{\text{deg}(\phi)}\int_\om \phi \wedge d\omega.
\end{align*}

\todo{Work out boundary terms, here the boundary integral is assumed zero (not always valid).}

Repeated integration by parts is then induced,
\begin{align*}
    \frac{dE}{du}\Big|_h &=
        \int_\Omega h\frac{\partial\mathcal{L}}{\partial u}
            + h^\prime\frac{\partial\fancyL}{\partial u^\prime}
            + \cdots
            + h^{(n)}\frac{\partial\fancyL}{\partial u^{(n)}}\,dx \\
    &= \sum_{k=0}^n \int_\om \frac{\partial\mathcal L}{\partial u^{(k)}} d^k h \,dx \\
    &= \sum_{k=0}^n \int_\om (-1)^k h d^k \frac{\partial\mathcal L}{\partial u^{(k)}}\,dx \\
    &= \int_\om \left(\sum_{k=0}^n (-1)^k d^k \frac{\partial\mathcal L}{\partial u^{(k)}}\right)h\,dx.
\end{align*}
It follows that
\begin{equation}\label{functional_gradient_zero_boundary}
    \frac{dE}{du} \coloneqq \sum_{k=0}^n (-1)^k d^k \frac{\partial\mathcal L}{\partial u^{(k)}}
\end{equation}
is a suitable definition. Setting this to zero gives the Euler-Lagrange equation of the functional $E$,
whose solutions are stationary points of $E$ with respect to variations.


% The formalisms of exterior calculus can be used to mechanistically derive integration-by-parts formulae.
% For example, in finite-element methods for Poisson equations, a generalization of the divergence theorem is used to remove the requirement
% of twice-weak-differentiability of the trial functions.
% \begin{align*}
%     & \int_\om \diver(\phi F)\, dx = \int_\om \star d(\phi \star F^\flat)\, dx = \int_\pom \phi F\cdot\hat{n}\, dx \\
%     \equiv\quad& \int_\om \star (d\phi \wedge \star F^\flat + \phi d\star F^\flat)\, dx = \int_\pom \phi F\cdot\hat{n}\, dx \\
%     \equiv\quad& \int_\om \nabla \phi \cdot F\, dx + \int_\om \phi\, \diver F\,dx = \int_\pom \phi F\cdot\hat{n}\, dx \\
% \end{align*}

% G\^ateaux derivative derivation of IBP for 1D. Geometric reasoning for divergence result.

\subsubsection{Some examples of Euler-Lagrange equations}
Here the Laplace equation and Plateau's problem for minimal surfaces are considered.
It is shown that the Laplace equation solves for a stationary function of the \textit{Dirichlet energy},
whose gradient descent algorithm leads to the heat equation.
\todo{example}

Applying the Euler-Lagrange equations to find stationary points of the arc-length/area functional of a curve/surface
leads to a non-linear diffusion process called mean-curvature flow, which can be implemented on a computer to find
locally \textit{minimal surfaces}, surfaces which interpolate a boundary and whose area can only increase under small perturbations.
\todo{example}
% >>>

\section{The finite element method}
\todo{
Discuss numerical solution of partial differential equations, central ideas in finite element methods.
I don't think a comparison to other methods would be very useful, should give a self-contained motivation for this particular
set of discretization ideas.
Describe connection to variational methods.
Galerkin methods.
}


\subsection{Weak forms of partial differential equations} % <<<
% Weak solutions are strong solutions if they are sufficiently differentiable.
% FEM has grounds in elliptic operator theory.
% Terminology, Dirichlet and Neumann, essential and natural, weak and strong, start to differ.
% The Riesz representation theorem.
% $a(.,.)$ induces the ``energy norm''.
% Galerkin methods
% Boundary conditions, Dirichlet Neumann Robin (p100 isogeometric analysis)
% Integration by parts
% Divergence theorem reasoning, integration by parts as generalization.
% Summation by parts, sequences.
% Weak derivatives
% Locally integrable functions
% $L^1_{loc}(\Omega)$ integrable over compact sets in the interior.
% Dirac delta function with weak derivative and integration by parts as motivation for generalised functions.
% Sobolev spaces
% $$\norm{f}_{w^k_p(\Omega)} \coloneqq \left(\sum_{|\alpha| \leq k} \norm{D_w^\alpha f}^p_{L^p(\Omega)}\right)^{\frac{1}{p}}.$$
% Does this just say that the weak derivatives are $L^p$ integrable? If a function is not $L^p$ integrable then it cannot be added to an $L^p$ integrable function to get an $L^p$ integrable function. $L^p$ is a vector space, so sums of $L^p$ functions are in $L^p$
% Might need H\"older's inequality.
% Cauchy-Schwartz inequality:
%     $$\left<u, v\right> \leq \sqrt{\left<u, u\right>}\sqrt{\left<v, v\right>}.$$
% Ciarlet finite element definition: Element domain, shape functions $P$, nodal basis $\mathcal{N}$ for $P^\pr = P^*$. The nodal
% basis can be of point-evaluation functionals. These can form a basis as the dual space is of the same finite dimension,
% and the values of point-evaluation functionals determine the shape function (and if there is a basis of shape functions, the coefficients of this basis).
% This is all in one element. Point nodal functionals will be along the boundary to give $C^0$ continuity. $(K,P,N)$ is a \textit{finite element}.
% A nodal basis of $P$ is defined as being dual to the nodal basis $\mathcal{N}$ of $P^*$. $\mathcal{N}_i(\phi_j) = \delta_{ij}$.
% A nodal basis is $1$ at the corresponding node point and zero at the others.
% Center-edge elements, Serendipity elements, Charged-particle point distribution


The partial differential equation (without initial or boundary conditions) is formulated as
\begin{align*}
    \fancyA u = f,
\end{align*}
where $\fancyA$ is a (not necessarily linear) differential operator.
The weak form of this PDE is
\begin{equation}\label{weak_form}
    \inner{\fancyA u, \phi} = \inner{f, \phi}
\end{equation}
for $\phi \in S \subset V$. If $S$ is finite dimensional with a basis of $\{\phi_1,\cdots,\phi_n\}$,
then a solution to \eqref{weak_form} is a solution to the system of equations
\begin{align*}
    \inner{\fancyA u, \phi_i} = \inner{f, \phi_i}
\end{align*}
for $1 \leq i \leq n$. Supposing $u_s = u^i \phi_i$ is an approximate solution, and $f_I$ is $f$ projected onto $S$, we can instead solve a finite system of equations
\begin{align*}
    & \inner{\fancyA u_s, \phi_i} = \inner{f, \phi_i} \\
    \equiv\quad& \inner{\fancyA(u^j \phi_j), \phi_i} = \inner{f_I, \phi_i}.
\end{align*}
This system can be expanded, even for non-linear $\fancyA$, into a system of equations in mass matrices.
For example, consider the non-linear equation with its weak form for test functions in $S$:
\begin{align*}
    \text{Strong form}\quad& -u^\ppr + u^2 = f \\
    \text{Weak form}\quad& \int_\Omega u^\pr \phi_i^\pr\, dx + \int_\Omega u^2 \phi_i\, dx = \inner{f, \phi_i}
\end{align*}
Considering again $u_s = u^i \phi_i$ and $f_I$ the projected $f$, the discrete weak form is
\begin{align*}
    todo
    % & u^j\int_\Omega \phi_j^\pr \phi_i^\pr\, dx + (u^j)^2\int_\Omega \phi_j^2 \phi_i\, dx = \inner{f_I, \phi_i}.
\end{align*}
--------
INCORRECT
--------
This is a non-linear system of equations that is feasibly solvable with a computer.
Letting $A = \left(\inner{\phi^\pr_i, \phi^\pr_j}\right)$, $B = \left(\inner{\phi_i^2, \phi_j}\right)$, $C = \left(\inner{\phi_i, \phi_j}\right)$, the equations are
\begin{align*}
    A\bar{u} + B\bar{u}^2 = C\bar{f},
\end{align*}
where $\bar{u}^2$ is squared component-wise, and $\bar{u}$, $\bar{f}$ are respective vectors of coefficients.

% >>>
% >>>
\subsection{Computing integrals over trial functions} % <<<

% Reference element
% Change of variables with Jacobian
% 
% Assembly of static data, tensors
% FEniCS form compiler

Let $G_T: T \rightarrow N \subset \mathbb{R}^n$ be defined as
    $$G_T(x) = p^i \Phi_i(x)$$
where $p^i \in \mathbb{R}^n$ are \textit{control points} and $\Phi_i$ are basis functions on the reference domain which form a partition of unity. This is a common paradigm for parametric surfaces,
such as B\'ezier patches, in geometric design.

\begin{align*}
    J = \pdf{G_T(x)_\gamma}{x_\tau} = p^i_\gamma\pdf{\Phi_i(x)_\gamma}{x_\tau}
\end{align*}

Consider a point-quadrature rule over points $g_i \in T$ with corresponding weights $w^i$.
The change of variables formula and quadrature is
\begin{align*}
    \int_N f(x)\, dx = \int_T f(x)|J(x)|\, dx \approx w^i f(g_i)|J(g_i)|.
\end{align*}
We can then reduce the amount of computation needed for these quadratures, if we know the basis functions over each element \textit{a priori},
by pre-computing a tensor of values
\begin{align*}
    \text{TABLE}(i, \gamma, \tau) = \pdf{\Phi_i(g_i)_\gamma}{x_\tau}.
\end{align*}
This stores all information about Jacobians of the element-mapping basis functions at the relevant
quadrature points, and thus forms a basis for the discrete Jacobian-field on the reference element.
The unknowns are the control points $p^i$. The quadrature is then computed as
\begin{align*}
    \int_N{f(x)\, dx} \approx w^i f(g_i) \left\vert\left(p^k_\gamma\,\text{TABLE}(k, \gamma, \tau)\right)_{\gamma,\tau}\right\vert.
\end{align*}

$dPhi(i, ..., \tau), dPsi, Phi(i, ...)$
``...'' are for vector-valued basis functions. $\tau$ is the axis index of the partial derivative.

\section{Solving Poisson's equation}

\section{Hyperbolic conservation laws}
\todo{
Advection, advection-diffusion, Burgers equation.
Derive analytic solution to advection equation with the method of characteristics.
Discuss propogation of information, the connection to hyperbolic conservation laws,
$$u_t + f_x = 0, f = f(u)$$
$$u_t + a(u)u_x = 0, a(u) = df/dt$$
Interpret as a total derivative
$$du/dt = \partial u /\partial t + \partial x / \partial t \partial u / \partial x = 0$$
Characteristics defined by IVPs
$$dx/dt = a(u).$$
}
\begin{equation}
    \pdf{u}{t} + u\pdf{u}{x} = 0.
\end{equation}
Using implicit Euler,
\begin{align*}
    &u^0 = g \\
    &\frac{u^{n+1} - u^n}{\Delta t} + u^{n+1}\pdf{u^{n+1}}{x} = 0.
\end{align*}
In the update equation, letting $u = u^{n+1}$ be the unknown and $u_p = u^n$ be constant, integrating against a test function gives
\begin{align*}
    &\int_\Omega \frac{u}{\Delta t}v + u\pdf{u}{x}v\,dx = \int_\Omega \frac{u_p}{\Delta t} v\,dx.
\end{align*}
Let $u = u^i\phi_i$ be a combination of basis trial functions. Since the left-hand-side is linear in the test function $v$, it is necessary and sufficient
that the equation is satisfied separately for each $v = \psi_j$. This gives a finite system of equations
\begin{align*}
    &\frac{u^i}{\Delta t}\int_\Omega \phi_i\psi_j\,dx + \int_{\Omega} (u^i\phi_i)(u^i \pdf{\phi_i}{x})\psi_j\,dx = \frac{1}{\Delta t}\int_\Omega u_p \psi_j\,dx.
\end{align*}


% >>>


\section{Deriving the Navier-Stokes equations}
\todo{Motivation}
\subsection{Conservation laws and continuity equations} % <<<
Let $\Omega \subset U$ be a domain in $U$. A conserved quantity, named here energy (measured in Joules), can advect and disperse, and this is modelled in general by
a flux function $j(x, t)$. For $U = \mathbb{R}^3$ this flux function has units $Jm^{-2}s^{-1}$.
It is assumed that energy is introduced into
the system purely through source term $p(x, t)$ with units $Jm^{-3}s^{-1}$.

The rate of change of total energy in $\Omega$ is then 
\begin{align*}
    \frac{d}{dt} \int_\Omega E\, dx = \int_{\partial\Omega} -j\cdot \hat{n}\, dS + \int_\Omega p\, dx,
\end{align*}
which by the divergence theorem is
\begin{align*}
    \int_\Omega \frac{\partial E}{\partial t} + \nabla \cdot j - p\, dx = 0.
\end{align*}
For this to be true of general domain $\Omega$, the integrand must be identically zero:
\begin{equation}
    \frac{\partial E}{\partial t} + \nabla \cdot j = p.
\end{equation}
This is a general form for a conservation law. If the flux $j$ is measured as the amount of energy being moved over a small boundary element
in the direction of the normal by a vector field $u$, then the vector field on the boundary of region $\Omega$ is transfering the quantity there either into, along, or out of the region. Therefore in this case $j = E u$ and the equation becomes
\begin{equation}
    \frac{\partial E}{\partial t} + \nabla \cdot (E u) = p.
\end{equation}
\subsubsection{Conservation of momentum}
For example, the conserved quantity may be \textit{momentum}, $\rho u$, the product
of scalar mass-density field $\rho$ and flow-velocity field $u$. As this is a conserved vector quantity, the conservation law is written
separately for each component of momentum, which is then compacted with $\otimes$ denoting the outer product:
\begin{equation}\label{ConservationOfMomentum}
\begin{split}
    \frac{\partial(\rho u_i)}{\partial t} + \nabla \cdot \left(\rho u_i u\right) &= p, \quad i=1,2,3\\
    \equiv\quad \frac{\partial(\rho u)}{\partial t} + \nabla \cdot \left(\rho u \otimes u\right) &= p.
\end{split}
\end{equation}
\subsubsection{Conservation of mass}
Equation \eqref{ConservationOfMomentum} is incomplete. Nothing has been said about restrictions on the mass density $\rho$.
The completing equation is
\begin{equation}\label{ConservationOfMass}
    \frac{\partial \rho}{\partial t} + \nabla \cdot (\rho u) = 0,
\end{equation}
which expresses that the only way mass enters or exits a closed region is by advection by $u$.
\todo{
Had ``If $u$ is conservative, mass is conserved''
which is wrong. Mass is always conserved.
A conservative vector field is not to do with this, rather to do with potentials.
}

\eqref{ConservationOfMomentum} and \eqref{ConservationOfMass} combined give the correct formulation, from first principles, of Newton's second law in continuum mechanics.
A more direct derivation, which makes the connection to Newton's second law much clearer, requires the material derivative.

\todo{
Really show why these two equations (conservation of mass and conservation of linear momentum) give
the Cauchy momentum equation.
}

% >>>
\subsection{Advection, transport, and the total and material derivatives} % <<<
Consider a vector field $u$ on a domain $U$. Any spatial quantity $\phi$ on $U$ can be transported along $u$.
The \textit{total derivative} is defined as
    $$\frac{d\phi}{dt} = \frac{\partial \phi}{\partial t} + \frac{\partial x}{\partial t} \cdot \nabla \phi.$$

When evaluated at $p$, this measures the rate of change of $\phi$ with respect to $t$-parameterised motion along a curve through $p$.
When this curve is given by the flow of a vector field $u$, this is called the \textit{material derivative}, denoted
    $$\frac{D\phi}{Dt} = \frac{\partial \phi}{\partial t} + u\cdot \nabla \phi.$$
Setting this to zero completes the derivation of the transport equation. $\frac{D \phi}{Dt} = 0$ (with $u$ implicit) states that from the point of view of
a particle moving along $u$, $\phi$ remains unchanged. $u$ can, for example, model the deformation velocity field of a material.
% >>>
\subsection{The Cauchy momentum equation} % <<<
In the 1820s Cauchy began the extension of Euler's laws of mechanics to deformable materials.
A fundamental object introduced is the \textit{Cauchy stress tensor}, denoted here by $T = T(x, t)$, which will now be motivated.

To distinguish continuum mechanics from point-particle mechanics, model-dependent internal forces are required. If these internal forces
were not present then this could hardly be called a cohesive object, and the mechanics would effectively reduce to that of a continuum of non-interacting point-particles.

By \textit{internal force} what is meant here is a force exerted on a small region of the material by its immediately adjacent particles.
Consider $\Omega \subset U$ a domain. Integrating the exerted forces along the boundary $\partial\Omega$ gives a vector flux, for which the divergence
theorem applies.
\begin{equation}\label{TDiv}
\begin{split}
    \int_{\partial\Omega} T \cdot \hat{n}\, dS
         = \int_\Omega \nabla \cdot T\, dV.
\end{split}
\end{equation}
The divergence theorem here gives a differential force, a measurement of internal stress for each point in the material.
Here $\nabla \cdot T$, when $U$ is three-dimensional, is understood to be the vector
    $$\nabla \cdot T = \bb \nabla \cdot T_1 & \nabla \cdot T_2 & \nabla \cdot T_3 \bbe^T,$$
where $T_i$ denote the stress in the unit normal aligned to axis $x_i$ (in matrix terms, the $i$'th column of $T$).



\todo{Geometric derivation of Cauchy stress tensor. (why is it a linear transformation?)}
% $T$ gives a means of measuring the internal forces of the material, and computationally takes the form of a (space-and-time-varying) matrix representing a linear transformation of oriented surface elements (given by normal vectors) to directional forces.


The \textit{Cauchy momentum equation} is
\begin{equation}\label{Cauchy}
    \rho \frac{Du}{Dt} = \nabla \cdot T + \rho g.
\end{equation}

$g = g(x, t)$ is a space-and-time-varying body force acting
on the system, such as, for example, Earth's gravitational pull.
This $g$ is a force, which is per-unit-mass, which is why
$\rho g$ appears in \eqref{Cauchy}.
\todo{---rederive this, maybe incorrect masses}

As defined above, $\frac{D}{Dt}$ denotes the \textit{material derivative}. In the mechanics of a point particle,
the particle has a velocity at each time. The definition of velocity and acceleration \textit{follow} this point.
This leads to the relevant idea of acceleration in continuum mechanics:
$\frac{Du}{Dt} = \frac{\partial u}{\partial t} + u\cdot \nabla u$.
By inertia we want to follow a particle as it continues in the direction of its velocity, and see if this velocity is changing.
This readily gives an interpretation of \eqref{Cauchy} as simply an instance of
\begin{equation}\label{Newton}
    F = ma.
\end{equation}
The left-hand-side of \eqref{Cauchy} is mass (here a differential density) times acceleration,
and the right-hand-side is force. Rather than a general force term, \eqref{Cauchy} splits the forces into internal stress forces
determined by the Cauchy stress tensor, and other external body forces. This is a general momentum equation for
continuum mechanics. A specific instance of a problem with a certain continuum material model requires
\begin{itemize}
    \item A definition of the Cauchy stress tensor, which likely relies on differential operators.
    \item A definition of any other external body forces acting on the material.
    \item Any other restrictive equations (for example, $\nabla \cdot u = r$).
    \item Initial/boundary conditions for which these combined equations form a well-posed problem.
\end{itemize}
% >>>
\subsection{The Euler equations for inviscid flow} % <<<
The Cauchy momentum equation is the basic skeleton for developing equations of fluid motion.

\subsubsection{Incompressibility}
We will assume that the fluid is \textit{incompressible}.
Compressibility is the basis of acoustic waves, so of course this assumption is non-physical. However, if we are interested in the bulk-movement flow,
in many cases acoustic waves will have neglible effect.

\subsubsection{Inviscid flow}
We need to define a Cauchy stress tensor. Internal forces are measured across an oriented surface element $\delta S$ with normal $\hat{n}$ by evaluating
$T\cdot \hat{n}$. The component of this force perpendicular to $\hat{n}$ is called a \textit{shear force}.
If there is no shear force, then the matrix representation of the Cauchy stress tensor will be $\alpha I$ for some scalar function $\alpha$.
The tensor divergence is equal to a gradient: $\nabla \cdot \left(\alpha I\right) = \nabla \alpha$. Letting $\alpha$ here be $\alpha = -p$,
the \textit{Euler equations of motion} are
\begin{equation}\label{Euler}
\begin{split}
    \rho \frac{Du}{Dt} = -\nabla p + \rho g \\
    \nabla \cdot u = 0.
\end{split}
\end{equation}
\todo{I think there might be some conservation terms missing}
The \textit{pressure} $p$ is an unknown. The Cauchy stress tensor, and thus the pressure, is \textit{defined} to give those internal non-shear forces
which give incompressibility.
% >>>
\subsection{Viscous flow and Newtonian fluids} % <<<
So far the derivations have been geometric rather than physical.
% >>>


\section{Solving (certain cases of) the Navier-Stokes equations}


\section{Appendix A: Classical mechanics}
\subsection{Potential and kinetic energy} % <<<
Consider the paths of particles in a closed system with position and momentum evolving in time
due to the action of a conservative force, $f = -\nabla U$ where $U$ is called the \textit{potential energy}.
Unit mass is assumed here.
\begin{equation}
\begin{split}
    x^\ppr = f(x) = -\nabla U(x).
\end{split}
\end{equation}

Given the conservative force field $f$, $U$ can be defined by quadrature as
\begin{equation}
\begin{split}
    U(x(t)) &= x_0 + \int_{t_0}^t x^\pr(\tau) \cdot(-f(x(\tau)))\, d\tau. \\
\end{split}
\end{equation}

We want to derive an energy conservation law $\frac{dE}{dt} = 0$, so as $U(x(t))$ varies, we need to define
some energy term $T$ where the potential energy ``comes from'' and ``goes'', and let $E = U + T$.
Potential energy $U$ is purely a function of position $x$. Assume that $T = T(x^\pr(t))$ is purely a function of momentum, with the assumption of unit mass.
We then need $\frac{dT}{dt} = -\frac{dU}{dt}$. By writing $T$ in terms of quadrature along the phase curve $(x(t),x^\pr(t))$, we have

\begin{equation}
\begin{split}
    T(x^\pr(t)) &= x_0^\pr + \int_{t_0}^t f(x(\tau)) \cdot \frac{dT}{dx^\pr}\, d\tau. \\
\end{split}
\end{equation}
Therefore we want $\frac{dT}{dx^\pr} = x^\pr$, and a possible energy term is $T = |x^\pr|^2/2$.
This, by construction, gives
\begin{equation}
\begin{split}
    \frac{d}{dt}\left(U + T\right) &= \left[x^\pr(t)\cdot (-f(x(t)))\right] + \left[f(x(t))\cdot x^\pr(t)\right] = 0.
\end{split}
\end{equation}
$T(x^\pr)$ is called the \textit{kinetic energy}. $E = U + T$ is called the \textit{total energy} of the configuration $(x(t), x^\pr(t))$ in the closed system.
% >>>





\begin{thebibliography}{9}
\bibitem{johann_bernoulli}
Johann Bernoulli, \textit{``Problema novum ad cujus solutionem Mathematici invitantur.'' (A new problem to whose solution mathematicians are invited.)}, 1696.
(retrieved from wikipedia/brachistochrone\_curve)

\bibitem{dirichlet_principle}
A. F. Monna, \textit{Dirichlet's principle: A mathematical comedy of errors and its influence on the development of analysis}, 1975


\bibitem{pde_larsson}
Stig Larsson, \textit{Partial differential equations with numerical methods}, 2003

\bibitem{lax_1973}
Peter Lax, \textit{Hyperbolic Systems of Conservation Laws and the Mathematical Theory of Shock Waves}, 1973


\end{thebibliography}

\end{document}
