% Terry Tao, on writing: https://terrytao.wordpress.com/advice-on-writing-papers/
% https://terrytao.wordpress.com/2007/03/18/why-global-regularity-for-navier-stokes-is-hard/#comment-625243
% \documentclass{article}
\documentclass[11pt,a4paper]{memoir}
% \setsecnumdepth{subsubsection}
\setsecnumdepth{subsection}
\setcounter{tocdepth}{2}
\setlrmarginsandblock{3cm}{3cm}{*} % Centre adjustment
\setulmarginsandblock{2.5cm}{*}{1}
\checkandfixthelayout 

\usepackage{amsmath}
\usepackage{amssymb}
\usepackage{mathtools}
\usepackage{microtype}
\usepackage{dirtytalk}
\usepackage{relsize}
\usepackage{csquotes}
\usepackage{epigraph}
\usepackage{physics}
\usepackage{cancel}
\usepackage{bm}
\newcommand{\bb}{\begin{bmatrix}}
\newcommand{\bbe}{\end{bmatrix}}
\newcommand{\pr}{\prime}
\newcommand{\ppr}{{\prime\prime}}
\newcommand{\pppr}{{\prime\prime\prime}}
\newcommand{\inner}[1]{\left<#1\right>}
\newcommand{\fancyA}{\mathcal{A}}
\newcommand{\fancyL}{\mathcal{L}}
\newcommand{\fancyN}{\mathcal{N}}
\newcommand{\fancyP}{\mathcal{P}}
% \newcommand{\norm}[1]{\left\Vert#1\right\Vert}
\newcommand{\om}{\Omega}
\newcommand{\pom}{{\partial\Omega}}
\newcommand{\omn}{\Omega_0}
\newcommand{\pomn}{{\partial\Omega_0}}
\newcommand{\diver}{\text{div}}
\newcommand{\Part}[2]{\frac{\partial #1}{\partial #2}}
% footnotes
\renewcommand\footnoterule{}

\newcommand{\todo}[1]{\vskip 0.1in \hrule \vskip 0.03in {#1} \vskip 0.03in \hrule \vskip 0.1in}

\title{The finite element method for the Navier-Stokes equations
\scriptsize{(working title)}
}
\author{Lucas Payne}

\begin{document}
\maketitle

\tableofcontents

\chapter{Continuum mechanics}
\chapterprecishere{
Corpus omne perseverare in statu suo quiescendi vel movendi uniformiter in directum, nisi quatenus a viribus impressis cogitur statum suum mutare.
\\\\
Mutationem motus proportionalem esse vi motrici impressae, \& fieri secundum lineam rectam qua vis illa imprimitur.
\\\\
Actioni contrariam semper \& aequalem esse reactionem: sive corporum duorum actiones in se mutuo semper esse aequales \& in partes contrarias dirigi.
\par\raggedleft \textup{Newton \cite{newton}}}
\section{Newton's laws of motion} % <<<

Newton's three laws, namely those of inertia, force, and equilibrium, have found universal success in application
to mechanical systems such as the pendulum, the motion of an rigid body, the evolution of a bending beam, and, as we shall see,
the motion of fluid. \textit{Mechanics} could be thought of as the study of physical motion, but the word ``physical'' might be misleading.
Newton's principles are mathematical in nature, applicable to the study of motion in a general sense as some unambiguously
measurable state which evolves in time.

\subsection{Symmetry, momenta, and inertia}
Mechanics as a theory of physical motion will require a definition of physical motion. A first attempt might be to posit
that ``physical states'' are representable as points in a finite-dimensional manifold, which we call the configuration space $C$, which is the case for typical
notions of state such as the two angles in a double pendulum, or the position and orientation of a rigid body. We might define a motion as
a continuous time-parameterized curve
    $$\gamma: [t_1, t_2] \rightarrow C.$$

(--- motivation of $F = ma$, and momenta as fundamental quantity).

We start in the middle:
\begin{equation}
    \text{Total force = change of momentum}.
\end{equation}
In this form, Newton's second law of motion states that a (non-explanatory) measurement of change of momentum will be called ``force''.



% >>>
\section{The Euler-Lagrange equations: \small{From $F = ma$ to $\Part{\fancyL}{q} - \frac{d}{dt}\Part{\fancyL}{\dot{q}} = 0$}} % <<<
\subsection{A Lagrangian of a mechanical system}
Force is an intensive measurement of the change in momentum. In the language of calculus,
\begin{equation}\label{force_equation}
    \int_{s_1}^{s_2} F\,dt = \fancyP(s_2) - \fancyP(s_1)
\end{equation}
for all time subintervals $[s_1, s_2] \subset [t_1, t_2]$. If we restrict $F$ to be conservative and a function only of position $q$, then we may
let $F = -\Part{V}{q}$ for some potential function $V$. Suppose also that $\mathcal{P} = \Part{T}{\dot{q}}$ for some potential function
(called the ``kinetic energy'') independent
of position $q$. We then define a Lagrangian of the mechanical system to be
    $$\fancyL(q, \dot{q}, t) = T(\dot{q}, t) - V(q, t) = \text{kinetic} - \text{potential}.$$
By definition, we have the force equation \eqref{force_equation} as
    $$-\int_{s_1}^{s_2} \Part{\fancyL}{q}\,dt = \Part{\fancyL}{\dot{q}}(s_2) - \Part{\fancyL}{\dot{q}}(s_1).$$
The step toward the calculus of variations (---)
\begin{equation}\label{el_inner_product_sum}
        \int_{s_1}^{s_2} \Part{\fancyL}{q}h\,dt + \int_{s_1}^{s_2} \Part{\fancyL}{\dot{q}}\frac{dh}{dt}\,dt = 0
    \quad\equiv\quad    \left<\Part{\fancyL}{q}, h\right> + \left<\Part{\fancyL}{\dot{q}}, \frac{dh}{dt}\right> = 0.
\end{equation}
Adjointness of differential operator $\frac{d}{dt}$ to $-\frac{d}{dt}$, and integration by parts. (---)
By linearity, we then get the reformulation of \eqref{el_inner_product_sum} as
\newcommand{\gateauxlagrangian}{\Part{\fancyL}{q} - \frac{d}{dt}\Part{\fancyL}{\dot{q}}}
    $$\left<\gateauxlagrangian, h\right> = 0$$
for all perturbation functions $h$.

\subsection{The first variation of a functional}
In the calculus of variations, $\gateauxlagrangian$ is an instance of the \textit{G\^ateaux derivative}, also called the ``first variation'' of
the functional
\begin{align*}
    S[q] \coloneqq \int_{t_1}^{t_2} \fancyL(q, \dot{q}, t)\,dt.
\end{align*}
The first variation measures response of the value of $S$, called the \textit{action}, to perturbations of the (differentiable) input function $q$,
and is denoted
\begin{equation}
    \frac{\delta S}{\delta q(t)} \coloneqq \gateauxlagrangian.
\end{equation}
The first variation is linear in the perturbation function, and so another term for the G\^ateaux derivative could be ``functional gradient''.
Setting this to zero gives the Euler-Lagrange equations, and the practice of determining trajectories of motions as stationary curves of the action is called the ``principle of stationary action''.

\subsection{From the Lagrangian to the equations of motion}
In the framework of Lagrangian mechanics, $\fancyP \coloneqq \Part{\fancyL}{\dot{q}}$ is the momentum. If there are $d$ degrees of freedom
in the mechanical system, and we suppose that $q_1,\cdots,q_d$ are (local) variables of state, then we say that $\fancyP_i \coloneqq \Part{\fancyL}{\dot{q}_i}$
are \textit{conjugate} to the $q_i$.

(---) By inducing the equations of motion by a Lagrangian, we get systems with a ``physical interpretation'' with ``physically meaningful'' force measurements.

\subsection{Why ---?}
These variational ideas appear because, no matter if our configuration space is finite-dimensional, time forms a continuum, and therefore if we
globally consider the calculus of the motion as a whole, it must be a ``variational'' calculus. (---)

(---) Physical process, conservation laws, conservative forces (hint at thermodynamics limiting the range of stress tensors?)

When we consider mechanical models of continuum processes, we will see that these same ideas appear in the spatial dimensions too.
A variational understanding of a continuum mechanics model leads very easily to a class of methods called Galerkin methods for solving
the corresponding (PDE) equations of motion.


% >>>
\section{Transport} % <<<

Before considering continuum processes in the framework of Newtonian and Lagrangian mechanics, we will look at a fundamental notion of a ``motion''
of a point in a function space. Many continuum models in physics, such as the heat equation,
Maxwell's equations, and the equations of fluid motion, are formed by \textit{continuity equations}. These laws posit that the
evolution of the state (represented by a function) is
due to the transport of the quantity that the function measures, which is either pushed around (by some flux either predetermined or dependent on the current state)
or created and destroyed at sources and sinks.

(--- figure of some manifold embedded in space, and some vector field pushing quantity around)

We consider here the transport of quantities (scalar, vector, and tensor) on a general finite-dimensional manifold $M$,
colloquially called ``the continuum''. All transporting vector fields (or flux functions) are considered to be tangent to this manifold $M$.

\subsection{Continuity equations and conservation laws}
\subsubsection{The integral form of a continuity equation}
Consider some spatial quantity $\phi$ on $M$ and a flux function $j$ which by which
this quantity flows around $M$. For clarity, we will begin by specializing $\phi$ to be a scalar, although later we will find it useful to
transport vector quantities such as momentum. By definition we want this flux function to just push quantity around, and not create or destroy it:
the creation and destruction of quantity is determined by some arbitrary source function $s$ (of the same kind as $\phi$). These variables are related by the continuity
condition
\begin{equation}\label{continuity_equation}
    \frac{d}{dt} \int_{\Omega_0} \phi\,dx = \int_{\Omega_0} s\,dx + \int_{\pomn} \phi j \cdot \left(-\hat{n}\right)\,dx
\end{equation}
for arbitrary control volumes $\Omega_0$ in the continuum. The term $-\hat{n}$ denotes the inward-pointing normal to the boundary of the control volume. This simply says that the change in the total quantity in the fixed control volume is accounted for exactly by that quantity pushed through the boundary by the flux function $j$, and the internal sources and sinks of quantity $s$.
--- $s_B$ as a boundary term? Why introduce this? Seems to be needed for the derivation of surface forces.
(--- It may be useful to add $s_G$ as a boundary source term at $\pomn \cap \pom$ so Dirac delta ideas don't need to be used for non-fluxed boundary source (or, for example, the domain might be a subdomain where the transported function is unknown outside, so the term is introduced through $s_G$.))

\subsubsection{The differential form of a continuity equation}
A common technique in continuum modelling is the use of Stokes' theorem to simplify integral expressions.
Equation \eqref{continuity_equation} becomes
\begin{equation}\label{continuity_equation_differential}
    \Part{\phi}{t} = s - \nabla\cdot (\phi j)
\end{equation}
assuming that $\phi j$ is sufficiently differentiable such that the limiting integral exists.
It should be noted that Stokes' theorem and its specializations are really \textit{definitions} of pointwise quantities
such as the divergence and curl as limits of these integral expressions for arbitrarily small regions.
Continuity relations are most naturally expressed in form \eqref{continuity_equation}, while the form
\eqref{continuity_equation_differential} may be more useful for techniques such as finite differences.
For example, it is a theorem of Gauss that in Euclidean space ($M = \mathbb{R}^3$) we have
\begin{equation}\label{gauss_euclidean_divergence}
    \nabla \cdot j = \Part{j_x}{x} + \Part{j_y}{y} + \Part{j_z}{z},
\end{equation}
and we get \eqref{continuity_equation_differential} in the form
\begin{equation}\label{continuity_equation_expanded}
    \Part{\phi}{t} = s - \nabla \phi \cdot j - \phi\left(\Part{j_x}{x} + \Part{j_y}{y} + \Part{j_z}{z}\right),
\end{equation}
by the product rule.
As one equation in a system of PDEs, \eqref{continuity_equation_expanded} is readily discretized by finite differences. For example, using forward difference in time and
central differences in space, our discrete scheme is
\begin{equation}\label{continuity_equation_finite_differences}
\begin{split}
    \frac{\phi(t + \Delta t) - \phi(t)}{\Delta t} = s &-\frac{\phi(\hat{x} + e_1\Delta x/2) - \phi(\hat{x} - e_1\Delta x/2)}{\Delta x}j_x \\
                                                      &-\frac{\phi(\hat{x} + e_2\Delta y/2) - \phi(\hat{x} - e_2\Delta y/2)}{\Delta y}j_y \\
                                                      &-\frac{\phi(\hat{x} + e_3\Delta z/2) - \phi(\hat{x} - e_3\Delta z/2)}{\Delta z}j_z \\
                                                      &-\frac{j_x(\hat{x} + e_1\Delta x/2) - j_x(\hat{x} - e_1\Delta x/2)}{\Delta x}\phi \\
                                                      &-\frac{j_y(\hat{x} + e_2\Delta y/2) - j_y(\hat{x} - e_2\Delta y/2)}{\Delta y}\phi \\
                                                      &-\frac{j_z(\hat{x} + e_3\Delta z/2) - j_z(\hat{x} - e_3\Delta z/2)}{\Delta z}\phi
\end{split}
\end{equation}
for $e_1,e_2,e_3$ the standard basis vectors in $\mathbb{R}^3$.
Later, when we discuss numerical methods for solving continuum models, we will not take this route. The methods of interest, \textit{Galerkin} methods,
work naturally with the integral form \eqref{continuity_equation}.
It will be seen later that some constructions in the presentation of Galerkin methods, such as the ``weak form'' of a PDE, simply undo the differentialization of the original integral form of physical PDEs.
(--- note: Maybe not exactly, as the integral conservation is quantified over regions, while the weak form is quantified over test functions,
which still need to be sufficiently differentiabile. But I think that this must express the same continuity relation.)

\subsection{The Reynolds transport theorem}
\subsubsection{The integral form of Reynolds transport}
With our integral formulation of a continuity relation \eqref{continuity_equation}, the control volume $\Omega_0$ is fixed.
We may change our perspective by considering, in addition to the flux function $j$ (which transports quantity $\phi$), another
vector field $\hat{u}$ which will transport our control volume $\Omega_0$. The rate of change of some time-dependent quantity $\gamma$ in this
\textit{moving} control volume is expressed as
\begin{equation}\label{reynolds_rate_of_change}
    \frac{d}{dt}\int_{\Omega_0(t)}\gamma\,dx,
\end{equation}
where $\Omega_0(t)$ implicitly denotes that $\Omega_0$ is being transported under the flow of $\hat{u}$.
Clearly, this rate of change of quantity $\gamma$ is due to the motion of the control volume (--- draw a picture of positive and negative contributions at
the boundary), as well as internal changes of $\gamma$ inside the (fixed) control volume.
The formal expression of these contributions to the rate of change \eqref{reynolds_rate_of_change} is
\begin{equation}\label{reynolds_transport_theorem}
    \eval{\frac{d}{dt}\left[\int_{\Omega_0(t)}\gamma\,dx\right]}_{t=0} =
        \int_{\Omega_0(0)}\Part{\gamma}{t}\,dx + \int_{\partial\Omega_0(0)}\gamma \hat{u}\cdot\hat{n} \,dx.
\end{equation}
This result is called the \textit{Reynolds transport theorem},
a generalization of Feynman's popularized ``differentiation under the integral sign'' \cite{feynman_trick},
otherwise named the Leibniz integral rule.
\subsubsection{The differential form of Reynolds transport}
 In the limit, with the routine application of Stokes' theorem, we can differentialize \eqref{reynolds_transport_theorem}
to get
\begin{equation}\label{reynolds_transport_theorem_differential}
    \frac{d_{\hat{u}}\gamma}{d_{\hat{u}}t} = \Part{\gamma}{t} + \nabla \cdot(\gamma \hat{u}),
\end{equation}
where $\frac{d_{\hat{u}}}{d_{\hat{u}}t}$ denotes a ``convective derivative'' with respect to $\hat{u}$, which measures
the change in volume of a quantity when a small control volume around the point of evaluation is moved, expanded or contracted by the flow field $\hat{u}$. (--- notation and terminology? This is not a material derivative, since the divergence of the flow field is considered to expand/contract the small control volume).

\subsubsection{Reynolds transport applied to a continuity equation}
Letting our quantity $\gamma$ in \eqref{reynolds_transport_theorem} be the quantity $\phi$ transported by flux function $j$
(described in continuity equation \eqref{continuity_equation}), we get a specialized form of the Reynolds transport theorem for continuity equations.
Term $\Part{\gamma}{t}$ in \eqref{reynolds_transport_theorem} becomes $\Part{\phi}{t}$ in the differential form of the continuity equation \eqref{continuity_equation_differential}, giving
\begin{equation}\label{reynolds_transport_continuity_equation}
\begin{split}
    \eval{\frac{d}{dt}\left[\int_{\Omega_0(t)}\phi\,dx\right]}_{t=0}
        &= \int_{\Omega_0(0)}-\nabla\cdot(\phi j) + s\,dx + \int_{\partial\Omega_0(0)}\phi \hat{u}\cdot\hat{n} \,dx \\
        &= \int_{\Omega_0(0)}s\,dx + \int_{\partial\Omega_0(0)}\phi (\hat{u} - j)\cdot\hat{n} \,dx
\end{split}
\end{equation}
by Stokes' theorem. This has a clear interpretation.
The $\hat{u} - j$ term is due to us wanting to measure the contributions to the total $\phi$ due to the moving boundary of
$\Omega_0$, where the motion that matters is \textit{relative} to the flux of the quantity $j$. Specifically, if we move the control volume by
the same flux function $j$ (letting $\hat{u} = j$), we get
\begin{equation}\label{lagrangian_transport}
    \eval{\frac{d}{dt}\left[\int_{\Omega_0(t)}\phi\,dx\right]}_{t=0}
        = \int_{\Omega_0(0)}s\,dx.
\end{equation}
In fact, \eqref{lagrangian_transport} is just another form for the conservation law \eqref{continuity_equation},
where the ``frame of reference'' for measurement of $\phi$ follows the transport of $\phi$. This simply means that as we follow some volume of quantity
original situated in $\Omega_0$, a conservation law posits that the only change detected is due to the source function $s$. In differential form
\eqref{lagrangian_transport} becomes
\begin{equation}\label{lagrangian_transport_differential}
    \frac{d_j\phi}{d_j t} = s,
\end{equation}
a succint equivalent to \eqref{continuity_equation_differential}.
The idea of following the flow while making measurements is called the \textit{Lagrangian} perspective, in contrast to the \textit{Eulerian} (fixed) perspective.
Before describing these notions of flow, we first investigate the property of incompressibility. (--- talk more later about Lagrangian versus Eulerian?)

\subsection{Incompressible and compressible transport}
% While the continuity equation \eqref{continuity_equation} gives us a general conservation law, transporting some quantity
% through our continuum (---without loss of generality, assume some domain $U \subset \mathbb{R}^2$), this is simply analogous
% to how general vector fields on the configuration space describe the evolution of a state in a finite-dimensional mechanical system
% (--- note that the ``vector field'' in consideration for the continuum case is really a ``vector field of vector fields'' on $C$, since
% a vector in the tangent space of $C$ here would be the vector field giving transport.
% Need to clearly distinguish these notions and use clear terminology, as it could be confusing).
Analogous to constraints on the motion of a finite mechanical system,
(--- section on Lagrangian mechanics should have an example of constraints of motion for a pendulum)
we can constrain possible movement of our continuous quantity to \textit{incompressible transport}. Much like how, in the framework of Lagrangian mechanics,
constraints on motion are implicitly enforced by strong ``virtual forces'', constraining transport to be non-compressing will lead to
the notion of \textit{pressure}, when we later consider the dynamics of the continuum.

\subsubsection{The material derivative}
(--- don't assume compressibility, todo. Define as a notion for a ``material point'' which flows but has no volume.)
Assuming an non-compressing (divergence-free) flux function $j$ which transports quantity $\phi$, the ``convective derivative'' in
\eqref{lagrangian_transport_differential} following a control volume becomes
\begin{equation}
    \frac{d_j\phi}{d_j t} = \Part{\phi}{t} + \nabla\cdot(\phi j) = \Part{\phi}{t} + j\cdot\nabla\phi + \cancel{\phi \nabla\cdot j}.
\end{equation}
We define the \textit{material derivative} to be
\begin{equation}\label{material_derivative}
    \frac{D}{Dt} \coloneqq \Part{}{t} + j\cdot\nabla.
\end{equation}
It is a convention to leave the vector field $j$ implicit, as material derivatives are usually taken with respect to some unambiguous velocity field.
(--- why define this?)

\subsubsection{Incompressibility}
Incompressibility of control volumes gives a constraint on the form of a flux function $j$.
We call this constrained flux function $j$ non-compressing. (---terminology? It makes more sense to call $j$ non-compressing rather than incompressible.)
By incompressibility we mean that a control volume being transported by $j$ will have
constant volume. While $j$ may transport other quantities, we express incompressibility by requiring the flux function to transport a constant ``volume quantity''
with a corresponding null source function,
    $$\phi_{\text{vol}} = 1,\quad s_{\text{vol}} = 0.$$
The corresponding conservation law, in differential form \eqref{continuity_equation_differential}, is
\begin{equation}\label{volume_conservation_law}
    \Part{\phi_{\text{vol}}}{t} = -\nabla \cdot (\phi_{\text{vol}}j) + s_{\text{vol}}
        \quad\Rightarrow\quad \nabla\cdot j = 0.
\end{equation}
This is our non-compressing constraint on $j$, and has a clear interpretation, as there is a non-zero divergence of $j$ if and only if
there is an inward or outward flux which would contract or expand a transported control volume.


\subsection{Transport of vector and tensor quantities}
All previous discussion on the transport of scalar quantities applies trivially to vector and tensor quantities.
This will soonest be of use in the discussion of conservation of linear momentum, a vector quantity (---should this be mentioned? It doesn't fit into the
framework so far as there is no notion of position map, this might be confusing and seem to imply ``linear momentum'' is natural for any continuum process).
However, some notational discussion is needed in order to establish differentialized forms of continuity equations and the Reynolds transport theorem.
\subsubsection{Reynolds transport of vector and tensor quantities}
For a general tensor quantity $\Gamma$, the integral form of Reynolds transport \eqref{reynolds_transport_theorem} is trivially
\begin{equation}\label{reynolds_transport_theorem_tensor}
    \eval{\frac{d}{dt}\left[\int_{\Omega_0(t)}\Gamma\,dx\right]}_{t=0} =
        \int_{\Omega_0(0)}\Part{\Gamma}{t}\,dx + \int_{\partial\Omega_0(0)}\Gamma \left(\hat{u}\cdot\hat{n}\right) \,dx.
\end{equation}
The step to the differential form \eqref{reynolds_transport_theorem_differential}, however, needs some thought
as rearranging
    $$\text{``}\Gamma\left(\hat{u}\cdot \hat{n}\right) = (\Gamma\hat{u})\cdot \hat{n}\text{''}$$
in order to apply the divergence theorem makes no sense. However, the divergence $\nabla \cdot$ was \textit{defined}
to evaluate the limit of this boundary integral for arbitrarily small $\Omega_0$. We therefore have a natural generalization of the
divergence for arbitrary tensors $\Psi$, as the limit of the boundary integral of the \textit{contraction} of $\Psi$ with the outward normal
$\hat{n}$ (which is a contravariant vector). The divergence of a rank $n$ tensor is then a rank $n-1$ tensor,
\begin{equation}\label{tensor_divergence}
    \int_{\Omega_0} \nabla\cdot\Psi\,dx \coloneqq
        \int_{\partial{\Omega_0}} \Psi : \hat{n}\,dx.
\end{equation}
We can then rewrite $\Gamma \left(\hat{u}\cdot \hat{n}\right)$ in \eqref{reynolds_transport_theorem_tensor} as
    $$\Gamma \left(\hat{u}\cdot \hat{n}\right) = \left(\Gamma \otimes \hat{u}\right) : \hat{n},$$
where the tensor product $\otimes$ ``defers contraction'' of $\hat{u}$ with $\hat{n}$, by storing it as a component of product tensor $\Gamma \otimes \hat{u}$.
This leads to a differentialization of \eqref{reynolds_transport_theorem_tensor},
\begin{equation}\label{reynolds_transport_theorem_tensor_differential}
    \frac{d_{\hat{u}}\Gamma}{d_{\hat{u}}t} = \Part{\Gamma}{t} + \nabla \cdot(\Gamma \otimes \hat{u}).
\end{equation}
(--- interpretation of this. This does actually make sense from ``first principles'' rather than tensor algebra.)
\subsubsection{Continuity equations for vector and tensor quantities}
With the previous ideas from tensor algebra, it will be easy to describe continuity relations for transport of tensors. The integral form of the scalar continuity equation \eqref{continuity_equation}, generalized to transported tensor $\Phi$, trivially becomes
\begin{equation}\label{continuity_equation_tensor}
    \frac{d}{dt} \int_{\Omega_0} \Phi\,dx = \int_{\partial\Omega_0} \Phi \left(j\cdot (-\hat{n})\right)\,dx + \int_{\Omega_0} s\,dx.
\end{equation}
By the same tensor algebra as above we have
    $$
        \Phi \left(j\cdot (-\hat{n})\right) = -\left(\Phi \otimes j\right) : \hat{n},
    $$
giving \eqref{continuity_equation_tensor} differentialized as
\begin{equation}\label{continuity_equation_tensor_differential}
    \Part{\Phi}{t} = -\nabla\cdot (\Phi \otimes j) + s.
\end{equation}
Finally, we may take a Lagrangian perspective on the transport of tensor $\Phi$ by letting the boundary transport in
\eqref{reynolds_transport_theorem_tensor_differential}
be the flux function, $\hat{u} = j$,
and $\Gamma$ be the tensor $\Phi$ being transported by $j$, giving
\begin{equation}\label{lagrangian_transport_tensor_differential}
    \frac{d_j \Phi}{d_j t} = \cancel{-\nabla\cdot\left(\Phi\otimes j\right)} + s + \cancel{\nabla\cdot\left(\Phi\otimes j\right)} = s.
\end{equation}
(--- This shows that tensor transport is a trivial modification of the previous results.)
\subsubsection{The meaning of tensor transport}
--- show that this is just a trivial notational convenience, as e.g. vector transport can just be done component-wise.
% >>>
\section{The kinematics of the continuum}
Transport equations are just one notion of ``physical motion'' in a continuum model.
These transport equations, with prescribed flux and source functions, determine a continuous process on a fixed
domain $M$. These conserved quantities (time-varying maps from $M$ to some measurement space (--scalars, tensors))
are then components of the total configuration space $C$, which clearly must be infinite-dimensional.
We now consider another, more geometrical component of $C$
which will let us model a physical domain with alterable shape.
In our discussion we will consider a fixed time interval $[t_1, t_2]$ in which our physical motions will take place.

\subsection{Position maps}
(--- terminology?)
We may consider the manifold $M$ as the parametric domain of some points living in an ambient manifold $N$.
We will call this the ``position map''
    $$x:M\times [t_1, t_2] \rightarrow N.$$
In general, $x$ needs not be continuous, differentiable, or invertible.
These restrictions are only introduced in accord to the physical meaning of the position map. For example, models of small beam deflections
may require continuity, and invertibility to prevent self-intersections.

(---figure of abstract square domain mapping to a bent beam, and a figure of a map from a reference configuration to itself.)

Continuum mechanics will study the physical motion of $x$, and other maps, where the equations of motion are, for example,
transport equations.


% \subsubsection{Displacement maps}
% --- possibly discuss.
% --- M is a subdomain of N, define a map which gives a displacement vector, which might only really make sense in Euclidean space.
% 
% \subsection{The configuration space of a continuum model}
% In sections (ref) and (ref) we discussed the mechanics of a physical system which can be encoded as a point in a finite-dimensional configuration manifold
% $C$. Clearly, however, if we consider $x$ a part of the physical state, the space of states will be infinite-dimensional. $x$ is just one possibly
% useful state variable which we have used for demonstration. We may, for example, consider continuum models for advection-diffusion processes \cite{turing} on some non-moving domain, without the use of a position function. $x$ is an instance of a \textit{point valued} map, in constrast to quantity (scalar or tensor) valued maps
% discussed previously. (---discuss why continuity equations would not apply (?))
% 
% Continuum mechanics will study the physical motion of $x$, and other maps, where the equations of motion are, for example,
% transport equations. Each component of our state will have a corresponding velocity. In the case of the position map $x : M \times [t_1, t_2] \rightarrow N$,
% the velocity is a vector field which
\subsection{Velocity}
Each component of our state $q \in C$ will have a corresponding velocity which ``generates'' a physical motion of that component.
In the case of the position map $x : M \times [t_1, t_2] \rightarrow N$, the velocity will be given by
a vector in the tangent space of $N$ at $x(r)$ for each parameter $r \in M$. (---it might be worth using more differential geometric language.)

(---draw this)

For some transported scalar quantity $\phi : M \times [t_1, t_2] \rightarrow \mathbb{R}$, the tangent space at each point of $\mathbb{R}$ is $\mathbb{R}$,
and therefore our velocity is represented by a scalar function giving local change of $\phi(r)$ for each $r \in M$.

(---draw this, the velocity as a small displacement function which is changing $\phi$)



\section{The dynamics of the continuum}

\subsection{Conservation of mass}
--- why?
\subsection{Conservation of linear momentum}

If we conserve the linear momentum $\rho u$, a ``quantity of motion'', under the flow of $u$, then we get a continuity equation
\begin{equation}\label{cauchy_continuity_traction}
    \eval{\frac{d}{dt}\left[\int_{\Omega_0(t)}\rho u\,dx\right]}_{t=0} = \int_{\omn(0)} \rho g\,dx + \int_{\pomn(0)}\hat{t}\,dx,
\end{equation}
a specific realization of the Lagrangian continuity equation \eqref{lagrangian_transport}. The Lagrangian perspective is convenient
as it allows us to factor out certain forces on a moving piece of material. The term $g$ is a regular body force per unit mass, where $\rho g$ corresponds to
the source term $s$ in \eqref{lagrangian_transport}. The boundary term involving $\hat{t}$, however, has no analogue in the scalar continuity equation
\eqref{lagrangian_transport}.
This vector term $\hat{t}$ is called the \textit{traction} in continuum mechanics, and measures a local force exerted across the boundary
of the control volume due to the immediately adjacent material.

\subsubsection{The Euler-Cauchy stress principle}
Clearly, in accord with Newton, we would like that two $\Omega_0$ and $\Omega_0^\prime$
which share a boundary element should have equal and opposite tractions across that boundary element.
Since the normal $\hat{n}$ represents a boundary element, and is negative for the opposite element, if $\hat{t}$ is linear in $\hat{n}$ we have this required
property. We can then let \eqref{cauchy_continuity_traction} become
\begin{equation}\label{cauchy_continuity}
    \eval{\frac{d}{dt}\left[\int_{\Omega_0(t)}\rho u\,dx\right]}_{t=0} = \int_{\omn(0)} \rho g\,dx + \int_{\pomn(0)}\sigma:\hat{n}\,dx
\end{equation}
where $\sigma$ is termed the \textit{Cauchy stress tensor}.


% <<<
\subsubsection{Differentializing the Cauchy momentum equation}
By application of the Reynolds transport theorem \eqref{reynolds_transport_theorem_tensor} to \eqref{cauchy_continuity} we get
\begin{equation}\label{cauchy_continuity_eulerian}
    \frac{d}{dt}\int_{\Omega_0}\rho u\,dx + \int_{\partial\Omega_0}\rho u (u\cdot \hat{n})\,dx = \int_{\Omega_0} \rho g\,dx + \int_{\pomn} \sigma:\hat{n}\,dx.
\end{equation}
Differentializing \eqref{cauchy_continuity_eulerian}, by our previously derived tensor identities, gives
\begin{equation}\label{cauchy_continuity_differential}
    \Part{(\rho u)}{t} + \nabla \cdot (\rho u\otimes u) = \rho g + \nabla\cdot\sigma.
\end{equation}
We can derive another more convenient form of \eqref{cauchy_continuity_differential} using the fact that
$\rho$ is conserved and has no source. Here, this will be derived purely algebraically, although the final form of the equation
has a useful interpretation. Expanding the partial derivative
$$
    \Part{(\rho u)}{t} = \rho\Part{u}{t} + u\Part{\rho}{t}
$$
is simple. The tensor divergence $\nabla \cdot (\rho u\otimes u)$ is defined such that
$$
    \int_{\Omega_0} \nabla \cdot (\rho u\otimes u)\,dx = \int_{\pomn} (\rho u\otimes u) : \hat{n}\,dx = \int_{\pomn} \rho u (u\cdot\hat{n})\,dx
$$
for arbitrary control volumes $\Omega_0$. As $\Omega_0$ becomes small, we can separately assume $u$ and $\rho u$ are constant
to derive
$$
    \int_{\pomn} \rho u (u\cdot\hat{n})\,dx = u\int_{\pomn}(\rho u)\cdot\hat{n}\,dx
                                              + \rho u\cdot \int_{\pomn} u\hat{n}\,dx \quad+\quad\cdots
$$
where a trailing term becomes neglible for a small control volume. This gives a ``tensor product rule'' for the divergence,
\begin{equation}\label{cauchy_divergence_tensor_product}
    \nabla\cdot (\rho u \otimes u) = u\nabla\cdot (\rho u) + \rho u\cdot\nabla u.
\end{equation}
Equation \eqref{cauchy_continuity_differential} then becomes
$$
    \rho\Part{u}{t} + u\Part{\rho}{t} + u\nabla\cdot (\rho u) + \rho u\cdot\nabla u = \rho g + \nabla\cdot\sigma.
$$
Noting that $\Part{\rho}{t}$ is already given by a continuity equation
$$\Part{\rho}{t} = -\nabla\cdot(\rho u),$$
as mass is transported by $u$ and has no source, we get
$$
    \rho\Part{u}{t} -\cancel{u\nabla(\rho u)} + \cancel{u\nabla\cdot (\rho u)} + \rho u\cdot\nabla u = \rho g + \nabla\cdot\sigma.
$$
Finally, the material derivative as defined in section (ref) is helpful in simplifying the above to
\begin{equation}\label{cauchy_continuity_differential_material}
    \rho\frac{Du}{Dt} = \rho g + \nabla\cdot\sigma.
\end{equation}
This form of \eqref{cauchy_continuity_differential} is more obviously a form of $F = ma$,
and is the usual presentation of the Cauchy momentum equation.
% From this derivation we have the identity
% \begin{equation}\label{mass_conserved_linear_momentum_identity}
%     \Part{(\rho u)}{t} + \nabla\cdot(\rho u\otimes u)
%     = \rho\frac{Du}{Dt}
% \end{equation}
% when mass density $\rho$ is conserved by $u$. The left-hand-side in \eqref{mass_conserved_linear_momentum_identity} is the differentialized Reynolds transport \eqref{reynolds_transport_theorem_tensor_differential},
% which measures the change in total linear momentum when following a small control volume that begins at a point $x^*$ and flows with $u$.
% If we integrate both sides over a control volume and apply the Reynolds transport theorem, we get
% \begin{equation}\label{mass_conserved_linear_momentum_identity_integral}
%     \eval{\frac{d}{dt}\left[\int_{\omn(t)} \rho u\,dx\right]}_{t=0} = \int_{\omn(0)} \rho \frac{Du}{Dt}\,dx.
% \end{equation}
% >>>
Recall that the material derivative is defined as
    $$\frac{D}{Dt} \coloneqq \Part{}{t} + u\cdot \nabla,$$
which measures the rate of change of a pointwise quantity from the perspective of a particle moving with the flow field $u$.
The equation \eqref{cauchy_continuity_differential_material} then says that, if the continuum consists of idealized points
each with a certain linear momentum (in the particle sense), deflection of their inertial path is due only to the application
of a body force $\rho g$ at this point, and a total traction force exerted by the surrounding material.


\subsection{Constitutive relations}
Body forces typically model
external fields such as gravity, which act on a material in bulk. We here forget the body forces, and investigate the possible forms of the
Cauchy stress tensor.
$\sigma$ models the ``material constitution'', and
therefore we call its specification a \textit{constitutive relation}.
A constitutive relation specifies how the kinematics of the continuum relate to its dynamics, as in, how the material configuration
induces forces on the material.

\subsubsection{Pressure in incompressible materials}
If we require the velocity field $u$ of the position map $x$ to be non-compressing (as described in section (ref)),
then we add to our model equations the constraint
\begin{equation}\label{noncompressing_velocity}
    \nabla \cdot u = 0.
\end{equation}
We proceed by analogy to a simple mechanical system.
The state of a pendulum of mass $m$ might be described by two spatial variables $X$ and $Y$ with a constraint
    $$X^2 + Y^2 = 1.$$
Given a differentiable pendulum motion, the linear momenta $mv_X$ and $mv_Y$ clearly cannot be constant, as then the pendulum
would ``fly out'' of its arc.
From the perspective of the $X,Y$ coordinate system there exists ``virtual forces'' which are exactly those that are needed
to maintain constraints. These forces have no necessary physical interpretation, as the pendulum model does not explicitly reference
the tension of the pendulum rod, but are rather those strong forces that must ``come from somewhere'' such that the constraints
are satisfied. In fact, if we change coordinates to $\theta$, no constraints and virtual forces are needed.
From a constraint on position $(X,Y)$ we can derive a constraint on velocity $(v_X,v_Y)$ by noting that the velocity must stay inside
the tangent plane in the configuration space. In this case, we require
\begin{equation}\label{pendulum_velocity_constraint}
    Xv_X + Yv_Y = 0.
\end{equation}
(--- add Lagrange multipliers to the system, then pressure here is a sort of Lagrange multiplier).
The continuum velocity constraint \eqref{noncompressing_velocity} is completely analogous to \eqref{pendulum_velocity_constraint}.
Motions of $x$ are restricted to those which do not compress mass. Clearly, our (per-point) linear momentum $\rho u$
cannot be constant except in simple parallel flows, implying the existence of some virtual (per-point) force.
We call this force the \textit{pressure} $p$, and we are now solving for $\rho$, $u$, and $p$ in a coupled system of equations
\eqref{cauchy_continuity_differential_material} and \eqref{noncompressing_velocity}, repeated here:
\begin{align*}
    \rho\frac{Du}{Dt} = \rho g + \nabla\cdot\sigma,\quad
        \nabla \cdot u = 0.
\end{align*}




\subsubsection{The Cauchy stress tensor}
\subsubsection{Body forces}

The linear momentum continuity equation \eqref{cauchy_continuity_differential_material}
can be modified by splitting the force term $F$ into ``surface'' and ``body'' forces, leading to the
\textit{Cauchy momentum equation}
\begin{equation}
    \rho \frac{Du}{Dt} = \nabla\cdot \sigma + \rho f.
\end{equation}


\subsubsection{Lagrangians of Cauchy momentum equations} % <<<
--- Require conservative body forces and ``thermodynamic conditions'' on surface forces, such that there exists a Lagrangian.
% >>>

\subsection{The continuum hypothesis and constitutive relations} % <<<
% >>>

\chapter{The Navier-Stokes equations}
% \section{The continuum hypothesis}
% \section{Body and stress forces}
% \section{Conservation laws}
% \subsection{Conservation of mass}
% \subsection{Conservation of linear momentum}
% \subsection{Conservation of angular momentum}
% \subsection{Conservation of energy}
% \subsection{Differentialization and coordinates}
% Navier-Stokes in expanded differential form for Cartesian coordinates. Maybe useful for finite differences.
% Working backwards to the weak form doesn't make much sense.
% \subsection{Kinds of fluids}
% \subsubsection{Incompressible fluids}
% \subsubsection{Inviscid flow}
% \subsubsection{Irrotational flow}
% \subsubsection{Steady flow}
% \subsubsection{Viscous flow and Newtonian fluids}

\chapter{The finite element method}
\section{Discretizing variational principles}
% The weak form is exactly analogous to perturbation principle in the principle of least action.
% To actually be an instance of the principle of least action, we need a Lagrangian. For example, the Poisson
% equation is the Euler-Lagrange equation of the Dirichlet energy.


\begin{thebibliography}{9}

\bibitem{newton}
Isaac Newton, \textit{Philosophiae Naturalis Principia Mathematica (Third edition)}, 1726.

\bibitem{johann_bernoulli}
Johann Bernoulli, \textit{``Problema novum ad cujus solutionem Mathematici invitantur.'' (A new problem to whose solution mathematicians are invited.)}, 1696.
(retrieved from wikipedia/brachistochrone\_curve)

\bibitem{dirichlet_principle}
A. F. Monna, \textit{Dirichlet's principle: A mathematical comedy of errors and its influence on the development of analysis}, 1975

\bibitem{pde_larsson}
Stig Larsson, \textit{Partial differential equations with numerical methods}, 2003

\bibitem{lax_1973}
Peter Lax, \textit{Hyperbolic Systems of Conservation Laws and the Mathematical Theory of Shock Waves}, 1973

\bibitem{lanczos}
Cornelius Lanczos, \textit{The Variational Principles of Mechanics}, 1952

\bibitem{batchelor}
G. K. Batchelor, \textit{Introduction to Fluid Dynamics}, 1967

\bibitem{leal}
L. Gary Leal, \textit{Advanced Transport Phenomena: Fluid Mechanics and Convective Transport Processes}, 2007

\bibitem{fem_ns}
Vivette Girault, Pierre-Arnaud Raviart, \textit{Finite Element Methods for Navier-Stokes equations}, 1986

\bibitem{feynman_trick}
Richard Feynman, \textit{Surely You're Joking, Mr. Feynman!}, 1985

\bibitem{arnold}
V.I. Arnol'd, \textit{Mathematical Methods of Classical Mechanics}, 1978

\bibitem{turing}
Alan Turing, \textit{The Chemical Basis of Morphogenesis}, 1952

\bibitem{applied_mathematics}
\textit{The Princeton Companion to Applied Mathematics}, 2015

\end{thebibliography}

\end{document}
